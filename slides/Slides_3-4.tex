%%%%%%%%%%%%%%%%%%%%% {{{
%%Options for presentations (in-class) and handouts (e.g. print).
\documentclass[pdf,9pt]{beamer}


%%%%%%%%%%%%%%%%%%%%%%
%Change this for different slides so it appears in bar
\usepackage{authoraftertitle}
\date{Chapter 3. Determinants and Diagonalization \\ \S 3-4. Application to Linear Recurrences}

%%%%%%%%%%%%%%%%%%%%%%
%% Upload common style file
\usepackage{LyryxLAWASlidesStyle}

\begin{document}

%%%%%%%%%%%%%%%%%%%%%%%
%% Title Page and Copyright Common to All Slides

%Title Page
\input frontmatter/titlepage.tex

%LOTS Page
\input frontmatter/lyryxopentexts.tex

%Copyright Page
\input frontmatter/copyright.tex

%%%%%%%%%%%%%%%%%%%%%%%%% }}}
\section[\textcolor{yellow}{}]{\textcolor{yellow}{Linear Recurrences}}
%-------------- start slide -------------------------------%{{{ 1
\frame{\frametitle{Linear Recurrences}
\begin{example}
The {\bf Fibonacci Numbers} are the numbers in the sequence
\[ 1, 1, 2, 3, 5, 8, 13, 21, 34, 55, 89, \ldots \]
and can be defined by the \alert{linear recurrence relation}
\[ f_{n+2} = f_{n+1} + f_{n}\mbox{ for all } n\geq 0,\]
with the initial conditions $f_0=1$ and $f_1=1$.
\end{example}

\uncover<2->{
\begin{problem}
Find $f_{100}$.
\end{problem}}

\uncover<3->{
Instead of using the recurrence to compute $f_{100}$, we'd
like to find a formula for $f_n$ that holds for all $n\geq 0$.}
}
%-------------- end slide -------------------------------%}}}
%-------------- start slide -------------------------------%{{{ 2
\frame{
\begin{definitions}
A sequence of numbers $x_0, x_1, x_2, x_3, \ldots $ is defined
\alert{recursively} if each number in the sequence is determined
by the numbers that occur before it in the sequence.

\uncover<2->{
A \alert{linear recurrence} of \alert{length $k$} has the
form
\[ x_{n+k} = a_1 x_{n+k-1} + a_2 x_{n+k-2} + \cdots + a_k x_{n},
n\geq 0, \]
for some real numbers $a_1, a_2, \ldots, a_k$.}
\end{definitions}
}
%-------------- end slide -------------------------------%}}}
%-------------- start slide -------------------------------%{{{ 3
\frame{
\begin{example}
The simplest linear recurrence has length one, so
has the form
\[ x_{n+1} = ax_{n}\mbox{ for } n\geq 0, \]
with $a\in\RR$ and some initial value $x_0$.

\uncover<2->{
In this case,
\begin{eqnarray*}
x_1 & = & ax_0 \\
x_2 & = & ax_1 = a^2 x_0 \\
x_3 & = & ax_2 = a^3 x_0 \\
\vdots & \vdots & \vdots \\
x_n & = & ax_{n-1} = a^n x_0
\end{eqnarray*}
Therefore, $x_n=a^nx_0$.
}
\end{example}
}
%-------------- end slide -------------------------------%}}}
%-------------- start slide -------------------------------%{{{ 4
\frame{
\begin{example}
Find a formula for $x_n$ if
\[ x_{n+2} = 2x_{n+1} +3x_{n}\mbox{ for } n\geq 0,\]
with $x_0=0$ and $x_1=1$.

\uncover<2->{
{\bf Solution.}
Define $V_n=\left[\begin{array}{c}
x_n \\ x_{n+1} \end{array}\right]$ for each $n\geq 0$.
Then
\[ V_0=\left[\begin{array}{c}
x_0 \\ x_{1} \end{array}\right]
=
\left[\begin{array}{cc}
0 \\ 1 \end{array}\right],\]
and for $n\geq 0$,
\[ V_{n+1}=\left[\begin{array}{c}
x_{n+1} \\ x_{n+2} \end{array}\right]
=
\left[\begin{array}{c}
x_{n+1} \\ 2x_{n+1}+3x_n \end{array}\right]
\]
}
\end{example}
}
%-------------- end slide -------------------------------%}}}
%-------------- start slide -------------------------------%{{{ 5
\frame{
\begin{example}[continued]
Now express $V_{n+1}=\left[\begin{array}{c}
x_{n+1} \\ 2x_{n+1}+3x_n \end{array}\right]$ as a matrix product:

\uncover<2->{
\[ V_{n+1}=\left[\begin{array}{c}
x_{n+1} \\ 2x_{n+1}+3x_n \end{array}\right]
=
\left[\begin{array}{cc}
0 & 1 \\
3 & 2
\end{array}\right]
\left[\begin{array}{c}
x_n \\ x_{n+1}
\end{array}\right]
=AV_n \]
}

\uncover<3->{
This is a linear dynamical system, so we can apply the
techniques from \S3.3, provided that $A$ is diagonalizable.}

\uncover<4->{
\[ c_A(x)=\det(xI-A)=
\left|\begin{array}{cc}
x & -1 \\
-3 & x-2
\end{array}\right|
= x^2-2x-3 = (x-3)(x+1)
\]
Therefore $A$ has eigenvalues $\lambda_1=3$ and $\lambda_2=-1$,
and \alert{is diagonalizable}.}
\end{example}
}
%-------------- end slide -------------------------------%}}}
%-------------- start slide -------------------------------%{{{ 6
\frame{
\begin{example}[continued]
$\vec{x}_1=\left[\begin{array}{c}
1 \\ 3 \end{array}\right]$
is a basic eigenvector corresponding to $\lambda_1=3$,
and
$\vec{x}_2=\left[\begin{array}{c}
-1 \\ 1 \end{array}\right]$
is a basic eigenvector corresponding to $\lambda_2=-1$.
\medskip

\uncover<2->{
Furthermore $P= \left[\begin{array}{cc}
\vec{x}_1 & \vec{x}_2 \end{array}\right]
=
\left[\begin{array}{cc}
1 & -1 \\
3 & 1
\end{array}\right]$
is invertible and is the diagonalizing matrix for $A$,
and
$P^{-1}AP=D=\left[\begin{array}{cc}
3 & 0 \\
0 & -1
\end{array}\right]$
}

\uncover<3->{
Writing $P^{-1}V_0 = \left[\begin{array}{c}
b_1 \\ b_2 \end{array}\right]$, we get
\[
\left[\begin{array}{c}
b_1 \\ b_2 \end{array}\right] =
\frac{1}{4}
\left[\begin{array}{cc}
1 & 1 \\
-3 & 1
\end{array}\right]
\left[\begin{array}{c}
0 \\
1
\end{array}\right]
=\left[\begin{array}{c}\vspace*{.02in}
\frac{1}{4} \\
\frac{1}{4}
\end{array}\right]
\]
}
\end{example}
}
%-------------- end slide -------------------------------%}}}
%-------------- start slide -------------------------------%{{{ 7
\frame{
\begin{example}[continued]
Therefore,
\begin{eqnarray*}
V_n =
\left[\begin{array}{c}
x_n \\ x_{n+1} \end{array}\right]
& = & b_1 \lambda_1^n \vec{x}_1 + b_2\lambda_2^n \vec{x}_2 \\
& = & \frac{1}{4}3^n
\left[\begin{array}{c}
1 \\ 3 \end{array}\right]
+ \frac{1}{4}(-1)^n
\left[\begin{array}{c}
-1 \\ 1 \end{array}\right],
\end{eqnarray*}
and so
\[ x_n = \frac{1}{4}3^n -\frac{1}{4}(-1)^n.\]
\end{example}
}
%-------------- end slide -------------------------------%}}}
%-------------- start slide -------------------------------%{{{ 8
\frame{
\begin{example}
Solve the recurrence relation
\[ x_{k+2}=5x_{k+1}-6x_k, k\geq 0\]
with $x_0=0$ and $x_1=1$.
\medskip

\uncover<2->{
{\bf Solution.}
Write
\[ V_{k+1}=\left[\begin{array}{c}
x_{k+1} \\ x_{k+2} \end{array}\right]
=
\left[\begin{array}{c}
x_{k+1} \\ 5x_{k+1}-6x_k \end{array}\right]
=
\left[\begin{array}{cc}
0 & 1 \\
-6 & 5
\end{array}\right]
\left[\begin{array}{c}
x_{k} \\ x_{k+1} \end{array}\right]
\]}

\uncover<3->{
Find the eigenvalues and corresponding eigenvectors for
\[ A = \left[\begin{array}{cc}
0 & 1 \\
-6 & 5
\end{array}\right] \]}
\end{example}
}
%-------------- end slide -------------------------------%}}}
%-------------- start slide -------------------------------%{{{ 9
\frame{
\begin{example}[continued]
$A$ has eigenvalues $\lambda_1=2$ with corresponding eigenvector
$\vec{x}_1=\left[\begin{array}{c}
1 \\ 2 \end{array}\right]$,
and
$\lambda_2=3$ with corresponding eigenvector
$\vec{x}_2=\left[\begin{array}{c}
1 \\ 3 \end{array}\right]$.

\uncover<2->{
\[ P = \left[\begin{array}{cc}
1 & 1 \\
2 & 3
\end{array}\right],
P^{-1} = \left[\begin{array}{cc}
3 & -1 \\
-2 & 1
\end{array}\right], \]
and
\[
\left[\begin{array}{c}
b_1 \\ b_2 \end{array}\right]
= P^{-1}V_0 =
\left[\begin{array}{cc}
3 & -1 \\
-2 & 1
\end{array}\right]
\left[\begin{array}{c}
0 \\ 1 \end{array}\right]
=
\left[\begin{array}{c}
-1 \\ 1 \end{array}\right] \] }

\uncover<3->{
Finally,
\[ V_k = \left[\begin{array}{c}
x_k \\ x_{k+1} \end{array}\right]
=
b_1\lambda_1^k\vec{x}_1 + b_2\lambda_2^k\vec{x}_2
= (-1)2^k
\left[\begin{array}{c}
1 \\ 2 \end{array}\right]
+ 3^k
\left[\begin{array}{c}
1 \\ 3 \end{array}\right]
\] }
\end{example}
}
%-------------- end slide -------------------------------%}}}
%-------------- start slide -------------------------------%{{{ 10
\frame{
\begin{example}
\[ \left[\begin{array}{c}
x_k \\ x_{k+1} \end{array}\right]
= (-1)2^k
\left[\begin{array}{c}
1 \\ 2 \end{array}\right]
+ 3^k
\left[\begin{array}{c}
1 \\ 3 \end{array}\right]
\]
and therefore
\[ x_k = 3^k - 2^k.\]
\end{example}
}
%-------------- end slide -------------------------------%}}}
\end{document}

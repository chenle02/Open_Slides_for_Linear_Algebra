%%%%%%%%%%%%%%%%%%%%% {{{
%%Options for presentations (in-class) and handouts (e.g. print).
% \documentclass[pdf,9pt]{beamer}
\documentclass[pdf,9pt]{beamer}


%%%%%%%%%%%%%%%%%%%%%%
%Change this for different slides so it appears in bar
\usepackage{authoraftertitle}
\date{Chapter 1. Systems of Linear Equations \\ \S  1-3. Homogeneous Equations}

%%%%%%%%%%%%%%%%%%%%%%
%% Upload common style file
\usepackage{LyryxLAWASlidesStyle}

\begin{document}

%%%%%%%%%%%%%%%%%%%%%%%
%% Title Page and Copyright Common to All Slides

%Title Page
\input frontmatter/titlepage.tex

%LOTS Page
\input frontmatter/lyryxopentexts.tex

%Copyright Page
\input frontmatter/copyright.tex

%%%%%%%%%%%%%%%%%%%%%%%%% }}}
%-------------- start slide -------------------------------%{{{ 2
\begin{frame}[fragile]
   \tableofcontents
\end{frame}
%-------------- end slide -------------------------------%}}}
\section[\textcolor{yellow}{}]{\textcolor{yellow}{Homogeneous Equations}}
%-------------- start slide -------------------------------%{{{ 3
\frame{
\frametitle{Homogeneous Equations}
\pause 
\begin{definition}
    A \alert{homogeneous linear equation} is one whose constant term
    is equal to zero.
    A system of linear equations is called \alert{homogeneous}
    \index{system of equations! homogeneous} \index{homogeneous system}
    if each equation in the system is homogeneous.
    A \alert{homogeneous system} has the form
    \begin{equation*}
	\left\{
	\begin{array}{ccccccccc}
	    a_{11}x_{1} & + & a_{12}x_{2} & + & \cdots & + & a_{1n}x_{n} & = & 0 \\
	    a_{21}x_{1} & + & a_{22}x_{2} & + & \cdots & + & a_{2n}x_{n} & = & 0 \\
			&   &             &   & \vdots &   &             &   &   \\
	    a_{m1}x_{1} & + & a_{m2}x_{2} & + & \cdots & + & a_{mn}x_{n} & = & 0
	\end{array}
	\right.
    \end{equation*}
    where $a_{ij}$ are scalars and $x_{i}$ are variables,
    $1\leq i\leq m$, $1\leq j\leq n$.
\end{definition}
\vfill
\pause

\begin{remark}
    \begin{enumerate}
        \item Notice that $x_1=0, x_2=0, \cdots, x_n=0$ is always a solution to a homogeneous system of equations. We call this the \alert{trivial solution}.
	    \vspace{1em}
	\item We are interested in finding, if possible, \alert{nontrivial solutions} (ones with at least one variable not equal to zero) to homogeneous systems.
    \end{enumerate}
\end{remark}
}
%-------------- end slide -------------------------------%}}}
%-------------- start slide -------------------------------%{{{ 4
\frame{
\begin{example}
Solve the system
$\left\{\begin{array}{ccccccccc}
    x_1  & + & x_2  & - & x_3  & + & 3x_4 & = & 0 \\
    -x_1 & + & 4x_2 & + & 5x_3 & - & 2x_4 & = & 0 \\
    x_1  & + & 6x_2 & + & 3x_3 & + & 4x_4 & = & 0
\end{array}\right.$
\end{example}

\pause
\begin{solution}
\[
    \left[\begin{array}{rrrr|r}
	1  & 1 & -1 & 3  & 0 \\
	-1 & 4 & 5  & -2 & 0 \\
	1  & 6 & 3  & 4  & 0
    \end{array}\right]
    \pause
    \rightarrow \cdots \rightarrow
    \left[\begin{array}{rrrr|r}
	1 & 0 & -9/5 & 14/5 & 0 \\
	0 & 1 & 4/5  & 1/5  & 0 \\
	0 & 0 & 0    & 0    & 0
    \end{array}\right]
\]
\pause
The system has infinitely many solutions, and the general
solution is
\[
    \left\{\begin{array}{ccl}
	x_1 & = & \vspace{0.05in}\frac{9}{5}s -\vspace{0.05in}\frac{14}{5}t \\
	x_2 & = & -\vspace{0.05in}\frac{4}{5}s -\vspace{0.05in}\frac{1}{5}t \\
	x_3 & = & s \\
	x_4 & = & t
    \end{array}\right. \pause ~\mbox{ or }~
    \left[\begin{array}{c}
	x_1 \\ x_2 \\ x_3 \\ x_4
    \end{array} \right]
    =
    \left[\begin{array}{c}
	\vspace{0.05in}\frac{9}{5}s -\vspace{0.05in}\frac{14}{5}t \\
	-\vspace{0.05in}\frac{4}{5}s -\vspace{0.05in}\frac{1}{5}t \\
	s \\
	t
    \end{array}\right],
    \forall s, t\in\RR.
\]
\end{solution}
}
%-------------- end slide -------------------------------%}}}
%-------------- start slide -------------------------------%{{{ 5
\frame{
 \begin{theorem}
     If a homogeneous system of linear equations has more variables than equations, then it has a nontrivial solution (in fact, infinitely many).
\end{theorem}

% \bigskip\pause

% \begin{problem}
%     We call the graph of an equation $ax^2+bxy+cy^2+dx+ey+ f = 0$ a conic
%     if the numbers $a$, $b$, and $c$ are not all zero.
%
%     \medskip
%
%     Show that there is at least one conic through any
%     five points in the plane that are not all on a line.
% \end{problem}

}
%-------------- end slide -------------------------------%}}}
\section[\textcolor{yellow}{}]{\textcolor{yellow}{Linear Combination}}
%-------------- start slide -------------------------------%{{{ 6
\frame{
\frametitle{Linear Combination}
\pause
\begin{definition}
    If $X_1, X_2, \ldots, X_p$ are columns with the same number
    of entries, and if $a_1, a_2,\ldots a_p \in\RR$ (are scalars)
    then $a_1X_1 + a_2X_2 + \cdots + a_pX_p$
    is a \alert{linear combination} of columns
    $X_1, X_2, \ldots, X_p$.
\end{definition}
\vfill
\pause
\begin{example}[continued]
    In the previous example,
    \begin{eqnarray*}
    {\left[\begin{array}{c}
    x_1 \\ x_2 \\ x_3 \\ x_4
    \end{array} \right]
    =
    \left[\begin{array}{c}
	\vspace{0.05in}\frac{9}{5}s -\vspace{0.05in}\frac{14}{5}t \\
	-\vspace{0.05in}\frac{4}{5}s -\vspace{0.05in}\frac{1}{5}t \\
	s \\
	t
    \end{array}\right]}
    \pause & = &
    {\left[\begin{array}{r}
	\vspace{0.05in}\frac{9}{5}s \\
	-\vspace{0.05in}\frac{4}{5}s \\
	s \\
	0
    \end{array}\right]
    +
    \left[\begin{array}{r}
    -\vspace{0.05in}\frac{14}{5}t \\
    -\vspace{0.05in}\frac{1}{5}t \\
    0 \\
    t
    \end{array}\right]} \\
    \pause & = &
    {s\left[\begin{array}{r}
     9/5 \\
    -4/5 \\
    1 \\
    0
    \end{array}\right]
    +
    t \left[\begin{array}{r}
    -14/5 \\
    -1/5 \\
    0 \\
    1
    \end{array}\right]}
    \end{eqnarray*}
\end{example}
}
%-------------- end slide -------------------------------%}}}
%-------------- start slide -------------------------------%{{{ 7
\frame{
\begin{example}[continued]
This gives us

\[
    \left[\begin{array}{c}
x_1 \\ x_2 \\ x_3 \\ x_4 \end{array} \right]
=
s\left[\begin{array}{r}
 9/5 \\
-4/5 \\
1 \\
0
\end{array}\right]
+
t \left[\begin{array}{r}
-14/5 \\
-1/5 \\
0 \\
1
\end{array}\right]
= sX_1 + tX_2,\quad
\]
\[\text{with}\quad
    X_1 = \left[\begin{array}{r}
 9/5 \\
-4/5 \\
1 \\
0
\end{array}\right]\quad\text{and}\quad
X_2=\left[\begin{array}{r}
-14/5 \\
-1/5 \\
0 \\
1
\end{array}\right].\]

\pause
\bigskip

The columns $X_1$ and $X_2$ are called
\alert{basic solutions} to the original homogeneous system.

\end{example}
}
%-------------- end slide -------------------------------%}}}
%-------------- start slide -------------------------------%{{{ 8
\frame{
\begin{example}[continued]
Notice that
\begin{eqnarray*}
{\left[\begin{array}{c}
x_1 \\ x_2 \\ x_3 \\ x_4 \end{array} \right]
=
s\left[\begin{array}{r}
 9/5 \\
-4/5 \\
1 \\ 0
\end{array}\right]
+
t \left[\begin{array}{r}
-14/5 \\
-1/5 \\
0 \\ 1
\end{array}\right]}
& = &
{\frac{s}{5}
\left[\begin{array}{r}
9 \\ -4 \\ 5 \\ 0
\end{array}\right]
+ \frac{t}{5}
\left[\begin{array}{r}
-14 \\ -1 \\ 0 \\ 5
\end{array}\right]} \\
& = &
{r \left[\begin{array}{r}
9 \\ -4 \\ 5 \\ 0
\end{array}\right]
+ q \left[\begin{array}{r}
-14 \\ -1 \\ 0 \\ 5
\end{array}\right]} \\
& = & r(5X_1) + q(5X_2)
\end{eqnarray*}
where $r, q\in\RR$.
\end{example}
}
%-------------- end slide -------------------------------%}}}
%-------------- start slide -------------------------------%{{{ 9
\frame{
\begin{example}[continued]
    The columns
    $5X_1 =\left[\begin{array}{r}
    9 \\ -4 \\ 5 \\ 0
    \end{array}\right]$ and
    $5X_2= \left[\begin{array}{r}
    -14 \\ -1 \\ 0 \\ 5
    \end{array}\right]$
    are also basic solutions to the original homogeneous
    system.
\end{example}
\vfill
\pause

\begin{remark}
    In general, any nonzero multiple of a basic solution (to a homogeneous system of linear equations) is also a basic solution.
\end{remark}
}
%-------------- end slide -------------------------------%}}}
%-------------- start slide -------------------------------%{{{ 10
\frame{
\begin{emptytitle}
    \alert{What does the rank tell us in the \textcolor{yellow}{homogeneous} case?}\\[1em]
    Suppose $A$ is the augmented matrix of an {\bf homogeneous} system
    of $m$ linear equations in $n$ variables, and $\rank A=r$.
    \[
    _{m}\left\{ \quad
    \left[\begin{array}{cccc|c}
	    * & * & * & * & \textcolor{yellow}{0} \\
	    * & * & * & * & \textcolor{yellow}{0} \\
	    * & * & * & * & \textcolor{yellow}{0} \\
	    * & * & * & * & \textcolor{yellow}{0} \\
	    * & * & * & * & \textcolor{yellow}{0}
    \end{array}\right]\right.
    \rightarrow
    \left[\begin{array}{cccc|c}
	\alert{1} & * & *         & *         & \textcolor{yellow}{0}  \\
	0         & 0 & \alert{1} & *         & \textcolor{yellow}{0}  \\
	0         & 0 & 0         & \alert{1} & \textcolor{yellow}{0}  \\
	0         & 0 & 0         & 0         & \textcolor{yellow}{0}  \\
	0         & 0 & 0         & 0         & \textcolor{yellow}{0}
    \end{array}\right]
    \]
    \vspace*{-.1in}
    % $\hspace*{1.10in} \underbrace{\hspace*{0.9in}}_n \;\;
    % \hspace*{.45in} \underbrace{\hspace*{0.9in}}_{r \; leading\; 1's}$
    $\hspace*{1.08in} \underbrace{\hspace*{0.75in}}_n \;\;
    \hspace*{.5in} \underbrace{\hspace*{0.8in}}_{r \; leading\; 1's}$
    \bigskip

    \pause
    There is always a solution, and the set of solutions to the system has $n-r$ parameters, so
    \vfill

    \pause
    \begin{itemize}
	\item if \textcolor{red}{$r<n$}, there is at least one parameter, and
    the system has infinitely many solutions;
    \pause
    \vfill
    \item if \textcolor{red}{$r=n$}, there are no parameters, and the system
    has a unique solution, the trivial solution.
    \end{itemize}
\end{emptytitle}
}
%-------------- end slide -------------------------------%}}}
%-------------- start slide -------------------------------%{{{ 11
\frame{
\begin{theorem}
    Let A be an $m \times n$ matrix of rank $r$, and consider the
    homogeneous system in $n$ variables with A as coefficient matrix. Then:
    \begin{enumerate}
	\item The system has exactly $n-r$ basic solutions, one for each parameter.
	\item Every solution is a \alert{linear combination} of these \alert{basic solutions}.
    \end{enumerate}
\end{theorem}

  % \pause
% \begin{proofnoend}
% Consider the RREF matrix equivalent to the augmented matrix of the system.
%
% Each non-leading variable corresponds to one of the $n-r$ parameters; let $N$ be the set of non-leading variables
% and enumerate the parameters as $s_j$, for $j\in N$.
% \pause
%
% Then, for scalars $c_{ij}$, the general solution has the form
% \begin{equation}\label{Eq.Row}
% x_i=\sum_{j\in N} c_{ij} s_j
% \end{equation}
% i.e. each variable---leading or not---is expressed as a linear combination of the parameters and there is no constant term.
% \end{proofnoend}
}
%-------------- end slide -------------------------------%}}}
%-------------- start slide -------------------------------%{{{ 12
\frame{
\begin{problem}
    Find all values of $a$ for which the system
    \[
	\left\{
	\begin{array}{ccccccc}
	    x & + & y  &   &    & = & 0 \\
	      &   & ay & + & z  & = & 0 \\
	    x & + & y  & + & az & = & 0
	\end{array}\right.
    \]
    has nontrivial solutions, and determine the solutions.
\end{problem}
\vfill
\pause
\begin{solution}
    Non-trivial solutions occur only when $a = 0$, and
    the solutions when $a=0$ are given by (rank $r=2$, $n-r=3-2=1$ parameter)
    \[ \left[ \begin{array}{r} x\\y\\z \end{array}\right] =
    s\left[\begin{array}{r} 1\\-1\\0 \end{array}\right],\quad\forall s\in\RR.\]
\end{solution}
}
%-------------- end slide -------------------------------%}}}
\end{document}

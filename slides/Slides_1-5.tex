%%%%%%%%%%%%%%%%%%%%% {{{
%%Options for presentations (in-class) and handouts (e.g. print).
\documentclass[pdf,9pt]{beamer}


%%%%%%%%%%%%%%%%%%%%%%
%Change this for different slides so it appears in bar
\usepackage{authoraftertitle}
\date{Chapter 1. Systems of Linear Equations \\ \S  1-5. Application to Electrical Networks}

%%%%%%%%%%%%%%%%%%%%%%
%% Upload common style file
\usepackage{LyryxLAWASlidesStyle}

\begin{document}

%%%%%%%%%%%%%%%%%%%%%%%
%% Title Page and Copyright Common to All Slides

%Title Page
\input frontmatter/titlepage.tex

%LOTS Page
\input frontmatter/lyryxopentexts.tex

%Copyright Page
\input frontmatter/copyright.tex

%%%%%%%%%%%%%%%%%%%%%%%%% }}}

\section[\textcolor{yellow}{}]{\textcolor{yellow}{Electrical Networks}}

%--------------start slide ---------------------% {{{ 1
\frame{\frametitle{Resistor Networks}
Important Symbols:
\begin{center}
Resistor: \begin{circuitikz} \draw (0,0) to [R] (2,0); \end {circuitikz}
\end{center}
\begin{center}
Voltage Source: \begin{circuitikz} \draw (0,0) to [/tikz/circuitikz/bipoles/length=0.75cm, battery1] (1,0); \end {circuitikz}
\end{center}
\begin{center}
Current:

\begin{circuitikz} \draw (0,0) node[scale=4]{$\circlearrowleft$}
(0,0) node{$I_1$}; \end{circuitikz}
\end{center}

\pause

\begin{emptytitle}
Resitance is measured in \em{ohms}, $\Omega$.
Voltage is measured in \em{volts}, $V$.
Current is measured in \em{amps}, $A$.
\end{emptytitle}
}
%------------end slide----------------------%}}}

%---------------start slide-------------------% {{{ 2
\frame{
\begin{problem}
Write an equation for each circuit and solve for each current in the following diagram.
\begin{center}
    \begin{circuitikz} \draw[scale=0.70]
	(0,4) to [battery1, v_= $17\; volts$] (0,0)
	to [R = $ 1 \Omega $] (4,0)
	to [R = $ 4 \Omega $] (4,4)
	(0,4) to [R =$ 2 \Omega $] (4,4)
	(4,4) to [battery1 = $14\; volts$] (8,4)
	(8,4) to [R = $2 \Omega$] (8,0)
	to [R = $2 \Omega$] (4,0)
	to [R = $3 \Omega$] (4,-4)
	(4,-4)to [R = $5 \Omega$] (0,-4)
	to [battery1 = $24\; volts$] (0,0)
	(2,2) node[scale=4]{$\circlearrowleft$}
	(2,2) node{$I_2$}
	(6,2) node[scale=4]{$\circlearrowleft$}
	(6,2) node{$I_3$}
	(2,-2) node[scale=4]{$\circlearrowleft$}
	(2,-2) node{$I_1$}
	;
    \end{circuitikz}
\end{center}

\end{problem}
}
%--------------end slide----------------------%}}}

%--------------start slide-------------------% {{{ 3
\frame{
\begin{solution}
The equation for the bottom circuit, with current $I_1$ is given by
\[
    5I_1 + 3I_1 + I_1 - I_2 = -24
\]
\pause
The top left circuit, with current $I_2$ is
\[
    I_2 - I_1 + 4I_2 - 4I_3 + 2I_2 = 17
\]
\pause
The top right circuit is
\[
    4I_3 - 4I_2 + 2I_3 + 2I_3 = -14
\]
\pause
After simplifying, this system is represented by
\[
    \left[
	\begin{array}{rrr|r}
	    9 & -1 & 0 & -24 \\
	    -1 & 7 & -4 & 17 \\
	    0 & -4 & 8 & -14
	\end{array}
    \right]
\]

\end{solution}
}
%---------------end slide-------------------%}}}

%--------------start slide-----------------% {{{ 4
\frame{
\begin{solution}[continued]
The reduced row-echelon form of this matrix is
\[
    \left[
	\begin{array}{rrr|r}
	    1 & 0 & 0 & -\frac{5}{2} \\
	    0 & 1 & 0 & \frac{3}{2} \\
	    0 & 0 & 1 & -1
    \end{array} \right]
\]
\pause
This gives values of the currents of
\begin{eqnarray*}
    I_1 &=& -\frac{5}{2} \\
    I_2 &=& \frac{3}{2} \\
    I_3 &=& -1
\end{eqnarray*}
\end{solution}
}
%----------------end slide-----------------%}}}

\end{document}

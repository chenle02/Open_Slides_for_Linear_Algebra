%%%%%%%%%%%%%%%%%%%%% {{{
%%Options for presentations (in-class) and handouts (e.g. print).
% \documentclass[pdf,9pt]{beamer}
\documentclass[pdf,9pt]{beamer}


%%%%%%%%%%%%%%%%%%%%%%
%Change this for different slides so it appears in bar
\usepackage{authoraftertitle}
\date{Chapter 1. Systems of Linear Equations \\ \S  1-2. Gaussian Elimination}

%%%%%%%%%%%%%%%%%%%%%%
%% Upload common style file
\usepackage{LyryxLAWASlidesStyle}

\begin{document}

%%%%%%%%%%%%%%%%%%%%%%%
%% Title Page and Copyright Common to All Slides

%Title Page
\input frontmatter/titlepage.tex

%LOTS Page
\input frontmatter/lyryxopentexts.tex

%Copyright Page
\input frontmatter/copyright.tex

%%%%%%%%%%%%%%%%%%%%%%%%% }}}
%-------------- start slide -------------------------------%{{{ 2
\begin{frame}[fragile]
   \tableofcontents
\end{frame}
%-------------- end slide -------------------------------%}}}
\section[\textcolor{yellow}{}]{\textcolor{yellow}{Row-Echelon Form}}
%-------------- start slide -------------------------------%{{{ 3
\frame{
\frametitle{Row-Echelon Matrix}
\pause
\begin{definition}
    A matrix is called a \textcolor{yellow}{row-echelon matrix} if
    \begin{itemize}
	\item All rows consisting entirely of zeros are at the bottom.
	\item The first nonzero entry in each nonzero row is a $1$ \\(called the {\bf leading 1} for that row).
	\item Each leading 1 is to the right of all leading $1$'s in rows above it.
    \end{itemize}
    A matrix is said to be in the \textcolor{yellow}{row-echelon form (REF)}  if it a row-echelon matrix.
\end{definition}
\vfill
\begin{example}
    \[
	\left[\begin{array}{rrrrrrrr}
		0          & \textcolor{pink}{1}   & *                     & *                     & *                     & *                     & *                     & * \\
		0          & 0                     & 0                     & \textcolor{pink}{1}   & *                     & *                     & *                     & * \\
		0          & 0                     & 0                     & 0                     & \textcolor{pink}{1}   & *                     & *                     & * \\
		0          & 0                     & 0                     & 0                     & 0                     & 0                     & 0                     & \textcolor{pink}{1} \\
     \textcolor{yellow}{0} & \textcolor{yellow}{0} & \textcolor{yellow}{0} & \textcolor{yellow}{0} & \textcolor{yellow}{0} & \textcolor{yellow}{0} & \textcolor{yellow}{0} & \textcolor{yellow}{0} \\
     \textcolor{yellow}{0} & \textcolor{yellow}{0} & \textcolor{yellow}{0} & \textcolor{yellow}{0} & \textcolor{yellow}{0} & \textcolor{yellow}{0} & \textcolor{yellow}{0} & \textcolor{yellow}{0}
	\end{array}\right]
    \]
    where $*$ can be any number.
\end{example}
}
%-------------- end slide -------------------------------%}}}
%-------------- start slide -------------------------------%{{{ 4
\frame{
\begin{definition}
    A matrix is called a \textcolor{yellow}{reduced row-echelon matrix} if
    \begin{itemize}
	\item Row-echelon matrix.
	\item Each leading 1 is the only nonzero entry in its column.
    \end{itemize}
    A matrix is said to be in the \textcolor{yellow}{reduced row-echelon form (RREF)} if it a reduced row-echelon matrix.
\end{definition}
\vfill
\pause
\begin{example}
    \[
    \left[\begin{array}{rrrrrrrr}
	0 & 1 & * & \textcolor{red}{0} & \textcolor{red}{0} & * & * & \textcolor{red}{0} \\
	0 & 0 & 0 & 1                  & \textcolor{red}{0} & * & * & \textcolor{red}{0} \\
	0 & 0 & 0 & 0                  & 1                  & * & * & \textcolor{red}{0} \\
	0 & 0 & 0 & 0                  & 0                  & 0 & 0 & 1 \\
	0 & 0 & 0 & 0                  & 0                  & 0 & 0 & 0 \\
	0 & 0 & 0 & 0                  & 0                  & 0 & 0 & 0
    \end{array}\right]
    \]
    where $*$ can be any number.
\end{example}
}
%-------------- end slide -------------------------------%}}}
%-------------- start slide -------------------------------%{{{ 5
\frame{

\begin{examples}
Which of the following matrices are in the REF?

\bigskip

Which ones are in the RREF?
\bigskip
\pause

(a) $\left[ \begin{array}{rrrr}  0 & 1 & 2 & 0\\ 0 & 0 & 1 & 2 \end{array}\right]$\hfill
(b) $\left[ \begin{array}{rrrr}   1 & 0 & 2 & 0\\ 0 & 0 & 1 & 2 \end{array}\right]$\hfill
(c) $\left[ \begin{array}{rrrr}  1 & 0 & 2 & 0\\ 0 & 0 & 1 & 2\\0 & 0 & 1 & 2 \end{array}\right]$
\bigskip
\bigskip

(d) $\left[ \begin{array}{rrrr}   1 & 0 & 2 & 0\\ 0 & 1 & 1 & 2 \end{array}\right]$\hfill
(e) $\left[ \begin{array}{rrrr}   1 & 2 & 0 & 0\\ 0 & 0 & 1 & 2 \end{array}\right]$\hfill
(f) $\left[ \begin{array}{rrrr}   1 & 2 & 0 & 0\\ 0 & 0 & 1 & 0\\0 & 0 & 0 & 1 \end{array}\right]$
\end{examples}

}
%-------------- end slide -------------------------------%}}}
%-------------- start slide -------------------------------%{{{ 6
\frame{
\begin{example}
Suppose that the following matrix is the augmented matrix
of a system of linear equations.
We see from this matrix that the system of linear equations has
four equations and seven variables.
\[
    \alert<2->{x_1}\hspace{1.4em}
    \textcolor<3->{teal}{x_2}\hspace{0.6em}
    \alert<2->{x_3}\hspace{1.1em}
    \alert<2->{x_4}\hspace{0.7em}
    \textcolor<3->{teal}{x_5}\hspace{1.1em}
    \textcolor<3->{teal}{x_6}\hspace{1.5em}
    \alert<2->{x_7}\hspace{2.5em}
\]
\[ \left[\begin{array}{rrrrrrr|r}
	\alert<2->{1} & \textcolor<3->{teal}{-3} & \alert<2->{4} & \alert<2->{-2} & \textcolor<3->{teal}{5 } & \textcolor<3->{teal}{-7} & \alert<2->{0}  & 4 \\
	0             & 0                         & \alert<2->{1} & \alert<2->{8}  & \textcolor<3->{teal}{0}  & \textcolor<3->{teal}{3 } & \alert<2->{-7} & 0 \\
	0             & 0                         & 0             & \alert<2->{1}  & \textcolor<3->{teal}{1}  & \textcolor<3->{teal}{-1} & \alert<2->{0}  & -1 \\
	0             & 0                         & 0             & 0             & 0                         & 0  & \alert<2>{1}  & 2
\end{array}\right] \]
Note that the matrix is a \textcolor{blue}{row-echelon matrix}.
\pause
\begin{itemize}
\item Each column of the matrix corresponds to a variable,
and the \alert{leading variables} are the variables that correspond
to columns containing leading ones.
\pause
\item The remaining variables are called \textcolor{teal}{non-leading variables}.  \end{itemize}
\end{example}
\pause
\begin{emptytitle}
We will use elementary row operations to transform a matrix to
row-echelon (REF) or reduced row-echelon form (RREF).
\end{emptytitle}
}
%-------------- end slide -------------------------------%}}}
\section[\textcolor{yellow}{}]{\textcolor{yellow}{Solving Systems of Linear Equations -- Gaussian Elimination}}
%-------------- start slide -------------------------------%{{{ 7
\frame{
\frametitle{Solving Systems of Linear Equations -- Gaussian Elimination}
\pause
\begin{theorem}
    Every matrix can be brought to (reduced) row-echelon form by a sequence of elementary row operations.
\end{theorem}
\pause
\bigskip
\begin{block}{\textbf{Gaussian  Elimination}}
To solve a system of linear equations proceed as follows:
\begin{enumerate}
    \item Carry the augmented matrix to a reduced row-echelon matrix using elementary row operations.  \pause
    \item If a row of the form \textcolor{teal}{$[0\;0\; \cdots 0\; | \; 1]$} occurs, the system is inconsistent.  \pause
    \item Otherwise assign the nonleading variables (if any) \alert{parameters} and use the equations corresponding to the reduced row-echelon matrix to solve for the leading variables in terms of the parameters.
\end{enumerate}
\end{block}
}
%-------------- end slide -------------------------------%}}}
%-------------- start slide -------------------------------%{{{ 8
\frame{
\shiftup{0.5}
\begin{problem}
    Solve the system
    \vspace*{-.28in}
    \[ \left\{\begin{array}{ccccccc}
	2x & + & y & + & 3z & = & 1 \\
	2y & - & z & + & x  & = & 0 \\
	9z & + & x & - & 4y & = & 2
    \end{array}\right.\]
\end{problem}
\vfill
\onslide<2->
\begin{solution}
    \[
	\hspace{-3em}\begin{array}{ crcr }
	&
    \left[\begin{array}{rrr|r}
	2 & 1  & 3  & 1 \\
	1 & 2  & -1 & 0 \\
	1 & -4 & 9  & 2
    \end{array}\right]&
    \onslide<3->
	\rightarrow^{r_1\leftrightarrow r_2}&
    \left[\begin{array}{rrr|r}
	1 & 2  & -1 & 0 \\
	2 & 1  & 3  & 1 \\
	1 & -4 & 9  & 2
    \end{array}\right]\\
    \\
    \onslide<4->
	    \rightarrow^{-2r_1+r_2, -r_1+r_3} &
    \left[\begin{array}{rrr|r}
	1 & 2  & -1 & 0 \\
	0 & -3 & 5  & 1 \\
	0 & -6 & 10 & 2
    \end{array}\right] &
    \onslide<5->
	    \rightarrow^{-2r_2+r_3}&
    \left[\begin{array}{rrr|r}
	1 & 2  & -1 & 0 \\
	0 & -3 & 5  & 1 \\
	0 & 0  & 0  & 0
    \end{array}\right]\\
    \\
    \onslide<6->
	    \rightarrow^{-\frac 13 r_2}&
    \left[\begin{array}{rrr|r}
	1 & 2 & -1   & 0    \\
	0 & 1 & -5/3 & -1/3 \\
	0 & 0 & 0    & 0
    \end{array}\right]&
    \onslide<7->
	    \rightarrow^{-2r_2+r_1}&
     \left[\begin{array}{rrr|r}
	1 & 0 & 7/3  & 2/3  \\
	0 & 1 & -5/3 & -1/3 \\
	0 & 0 & 0    & 0
    \end{array}\right]
	\end{array}
    \]
\end{solution}
}
%-------------- end slide -------------------------------%}}}
%-------------- start slide -------------------------------%{{{ 9
\frame{
\begin{solution}[continued]
Given the reduced row-echelon matrix
\[
 \left[\begin{array}{rrr|r}
1 & 0 & 7/3 & 2/3 \\
0 & 1 & -5/3 & -1/3 \\
0 & 0 & 0 & 0
\end{array}\right]
\]
$x$ and $y$ are \alert{leading variables}; $z$ is a
\alert{non-leading variable} and so assign a
\alert{parameter} to $z$.
\pause
Thus the solution to the original system is given by
\[
\left. \begin{array}{rrrrr}
x & = & \vspace{0.05in}\frac{2}{3} & - & \vspace{0.05in}\frac{7}{3}s\\
y & = & -\vspace{0.05in}\frac{1}{3}&  + & \vspace{0.05in}\frac{5}{3}s\\
z & = & s & & \\
\end{array} \right\} \mbox{ for all }s\in\RR.
\]
\end{solution}
}
%-------------- end slide -------------------------------%}}}
%-------------- start slide -------------------------------%{{{ 10
\frame{
\shiftup{0.5}
\begin{problem}
Solve the system
\vspace{-3em}
\[ \left\{\begin{array}{rrrrrrr}
x & + & y & + & 2z & = & -1 \\
y & + & 2x & + & 3z & = & 0 \\
z & - & 2y &  &  & = & 2
\end{array}\right.\]
\end{problem}
\vfill
\pause

{\begin{solution}
\[
    \begin{array}{crcr}
&
\left[\begin{array}{rrr|r}
1 & 1 & 2 & -1 \\
2 & 1 & 3 & 0 \\
0 & -2 & 1 & 2
\end{array}\right]
    &\rightarrow^{-2r_1+r_2}&
\pause
\left[\begin{array}{rrr|r}
1 & 1 & 2 & -1 \\
0 & -1 & -1 & 2 \\
0 & -2 & 1 & 2
\end{array}\right]\\
\\
% \vspace{1em}
	\rightarrow^{-1\cdot r_2}&
\pause
\left[\begin{array}{rrr|r}
1 & 1 & 2 & -1 \\
0 & 1 & 1 & -2 \\
0 & -2 & 1 & 2
\end{array}\right]&
\pause
	\rightarrow^{2r_2+r_3}&
\left[\begin{array}{rrr|r}
1 & 0 & 1 & 1 \\
0 & 1 & 1 & -2 \\
0 & 0 & 3 & -2
\end{array}\right]\\
% \vspace{0.5em}
\\
	\rightarrow^{\frac 13 r_3}&
\pause
\left[\begin{array}{rrr|r}
1 & 0 & 1 & 1 \\
0 & 1 & 1 & -2 \\
0 & 0 & 1 & -2/3
\end{array}\right]&
	\rightarrow^{-r_3+r_2, -r_3+r_1}&
\pause
\left[\begin{array}{rrr|r}
1 & 0 & 0 & 5/3 \\
0 & 1 & 0 & -4/3 \\
0 & 0 & 1 & -2/3
\end{array}\right]
    \end{array}
\]
%\shiftup{0.5}
\pause

{The \alert{unique} solution is
$x=5/3$, $y=-4/3$, $z=-2/3$.
\medskip
\pause

{\bf Check your answer!} }
\end{solution}}
}
%-------------- end slide -------------------------------%}}}
%-------------- start slide -------------------------------%{{{ 11
\frame{
\shiftup{0.5}
\begin{problem}
Solve the system
\vspace{-3em}
\[ \left\{
    \begin{array}{rrrrrrr}
	-3x_1 & - & 9x_2 & + & x_3 & = & -9 \\
	2x_1  & + & 6x_2 & - & x_3 & = & 6 \\
	x_1   & + & 3x_2 & - & x_3 & = & 2
    \end{array}\right.
\]
\end{problem}
\pause
\vfill

\begin{solution}
\[
\left[\begin{array}{rrr|r}
1 & 3 & -1 & 2 \\
2 & 6 & -1 & 6 \\
-3 & -9 & 1 & -9
\end{array}\right] \rightarrow
\left[\begin{array}{rrr|r}
1 & 3 & -1 & 2 \\
0 & 0 & 1 & 2 \\
0 & 0 & -2 & -3
\end{array}\right] \rightarrow
\left[\begin{array}{rrr|r}
1 & 3 & 0 & 4 \\
0 & 0 & 1 & 2 \\
0 & 0 & 0 & 1
\end{array}\right]
\]
\shiftup{0.5}
\pause

{The last row of the final matrix corresponds to the equation
%\shiftup{0.5}
\[ 0x_1 + 0x_2 + 0x_3= 1\]
%\shiftup{0.5}
which is impossible! \\[1em]

Therefore, this system is inconsistent, i.e., it has no solutions.}
\end{solution}
}
%-------------- end slide -------------------------------%}}}
%-------------- start slide -------------------------------%{{{ 12
\frame{
\begin{problem}[ General Patterns for Systems of Linear Equations ]
    Find all values of $a$, $b$ and $c$
    (or conditions on $a$, $b$ and $c$) so that the system
    \[ \begin{array}{rrrrrrr}
	2x & + & 3y & + & \alert<2->{a}z & = & \alert<2->{b} \\
	   & - & y  & + & 2z             & = & \alert<2->{c} \\
	 x & + & 3y & - & 2z             & = & 1
     \end{array} \]
    has (i) a unique solution, (ii) no solutions, and (iii) infinitely
    many solutions. In (i) and (iii), find the solution(s). \\
\end{problem}
\vfill
\pause
\begin{solution}
    \[ \left[\begin{array}{rrr|r}
	2 & 3  & a  & b \\
	0 & -1 & 2  & c \\
	1 & 3  & -2 & 1 \\
    \end{array}\right]
    \onslide<3->
    \rightarrow
    \left[\begin{array}{rrr|r}
	1 & 3  & -2 & 1 \\
	0 & -1 & 2  & c \\
	2 & 3  & a  & b \\
    \end{array}\right]
    \]
\end{solution}
}
%-------------- end slide -------------------------------%}}}
%-------------- start slide -------------------------------%{{{ 13
\frame{
\begin{solution}[continued]
    \[ \left[\begin{array}{rrr|r}
	1 & 3  & -2 & 1 \\
	0 & -1 & 2  & c \\
	2 & 3  & a  & b \\
    \end{array}\right]
    \onslide<2->
    \rightarrow
    \left[\begin{array}{rrr|r}
	1 & 3  & -2  & 1   \\
	0 & -1 & 2   & c   \\
	0 & -3 & a+4 & b-2 \\
    \end{array}\right]
    \]
    \onslide<3->
    \[ \rightarrow \left[\begin{array}{rrr|c}
	1 & 3  & -2  & 1   \\
	0 & 1  & -2  & -c  \\
	0 & -3 & a+4 & b-2 \\
    \end{array}\right]
    \onslide<4->\rightarrow
    \left[\begin{array}{rrr|c}
	1 & 0 & 4   & 1+3c   \\
	0 & 1 & -2  & -c     \\
	0 & 0 & a-2 & b-2-3c \\
    \end{array}\right]
    \]
    \onslide<5->{
    {\bf Case 1.} $a-2\neq 0$, i.e., $a\neq 2$.}
    \onslide<6->
    In this case,
    \shiftup{0.5}
    \[ \rightarrow\left[\begin{array}{rrr|c}
	1 & 0 & 4  & 1+3c               \\
	0 & 1 & -2 & -c                 \\
	0 & 0 & 1  & \frac{b-2-3c}{a-2} \\
    \end{array}\right]
    \onslide<7->\rightarrow
    \left[\begin{array}{rrr|c}
	1 & 0 & 0 & 1+3c-4\left(\frac{b-2-3c}{a-2}\right) \\
	0 & 1 & 0 & -c + 2\left(\frac{b-2-3c}{a-2}\right) \\
	0 & 0 & 1 & \frac{b-2-3c}{a-2}                    \\
    \end{array}\right]
    \]
\end{solution}
}
%-------------- end slide -------------------------------%}}}
%-------------- start slide -------------------------------%{{{ 14
\frame{
\begin{solution}[continued]
\[ \left[\begin{array}{ccc|c}
    1 & 0 & 0 & 1+3c-4\left(\frac{b-2-3c}{a-2}\right) \\
    0 & 1 & 0 & -c + 2\left(\frac{b-2-3c}{a-2}\right) \\
    0 & 0 & 1 & \frac{b-2-3c}{a-2}
\end{array}\right]
\]
\pause
(i) When $a\neq 2$, the unique solution is
\begin{align*}
    x & =1+3c-4\left(\frac{b-2-3c}{a-2}\right) \\
    y & =-c+2\left(\frac{b-2-3c}{a-2}\right)   \\
    z & =\frac{b-2-3c}{a-2}
\end{align*}
\end{solution}}
%-------------- end slide -------------------------------%}}}
%-------------- start slide -------------------------------%{{{ 15
\frame{
\begin{solution}[continued]
    {\bf Case 2.} If $a=2$, then the augmented matrix becomes
    \[
    \left[\begin{array}{ccc|c}
	1 & 0 & 4   & 1+3c \\
	0 & 1 & -2  & -c   \\
	0 & 0 & a-2 & b-2-3c
    \end{array}\right]
    \onslide<2->
    \rightarrow
     \left[\begin{array}{ccc|c}
	1 & 0 & 4  & 1+3c   \\
	0 & 1 & -2 & -c     \\
	0 & 0 & 0  & b-2-3c \\
    \end{array}\right]
    \]
    \pause

    From this we see that the system has no solutions when
    $b-2-3c\neq 0$.
    \medskip

    \pause
    (ii) When $a=2$ and $b-3c\neq 2$, the system has no solutions.
\end{solution}
}
%-------------- end slide -------------------------------%}}}
%-------------- start slide -------------------------------%{{{ 16
\frame{
\begin{solution}[continued]
    Finally when $a=2$ and $b-3c=2$, the augmented matrix becomes
    \pause{
    \[
    \left[\begin{array}{ccc|c}
    1 & 0 & 4 & 1+3c \\
    0 & 1 & -2 & -c \\
    0 & 0 & 0 & b-2-3c
    \end{array}\right]
    \rightarrow
    \left[\begin{array}{rrr|c}
    1 & 0 & 4 & 1+3c \\
    0 & 1 & -2 & -c \\
    0 & 0 & 0 & 0
    \end{array}\right]
    \]}

    \pause{
    and the system has infinitely many solutions.}

    \medskip
    \pause
    (iii) When $a=2$ and $b-3c=2$, the system has
    infinitely many solutions:
    \[
	\left.\begin{array}{rrrrr}
	x & = & 1+3c & - & 4s \\
	y & = & -c   & + & 2s \\
	z & = & s
	\end{array}\quad\right\}\quad \text{for all $s\in\RR$.}\]
   \myQED
\end{solution}
}
%-------------- end slide -------------------------------%}}}
\section[\textcolor{yellow}{}]{\textcolor{yellow}{Rank}}
%-------------- start slide -------------------------------%{{{ 17
\frame{
\frametitle{Rank}
\pause
\begin{definition}
The \alert{rank} of a matrix $A$, denoted $\rank A$, is the
number of leading 1's in any row-echelon matrix
obtained from $A$ by performing elementary row operations.
\end{definition}
}
%-------------- end slide -------------------------------%}}}
%-------------- start slide -------------------------------%{{{ 18
\frame{
\begin{emptytitle}
    Suppose $A$ is the augmented matrix of a {\bf consistent} system
    of $m$ linear equations in $n$ variables, and $\rank A=r$.
\end{emptytitle}
    \[
_{m}\left\{
\left[\begin{array}{cccc|c}
    * & * & * & * & * \\
    * & * & * & * & * \\
    * & * & * & * & * \\
    * & * & * & * & * \\
    * & * & * & * & *
\end{array}\right]\right.
\rightarrow
\left[\begin{array}{cccc|c}
    \textcolor{red}{1 } & * & *                  & *                  & * \\
    0                   & 0 & \textcolor{red}{1} & *                  & * \\
    0                   & 0 & 0                  & \textcolor{red}{1} & * \\
    0                   & 0 & 0                  & 0                  & 0 \\
    0                   & 0 & 0                  & 0                  & 0
\end{array}\right]
\]
\vspace*{-.1in}
$\hspace*{1.05in} \underbrace{\hspace*{0.75in}}_n \;\;
\hspace*{.5in} \underbrace{\hspace*{0.8in}}_{r \; leading\; 1's}$
\vfill

\pause
Then the set of solutions to the system has $n-r$ parameters, so

\pause
\begin{itemize}
    \item if \textcolor{red}{$r<n$}, there is at least one parameter, and the system has infinitely many solutions; \pause
    \item if \textcolor{red}{$r=n$}, there are no parameters, and the system has a unique solution.
\end{itemize}
}
%-------------- end slide -------------------------------%}}}
%-------------- start slide -------------------------------%{{{ 19
\frame{
\begin{problem}
    Find the rank of
    $A= \left[\begin{array}{rrr}
    a & b & 5 \\
    1 & -2 & 1 \end{array}\right]$.
\end{problem}
\pause

\begin{solution}
\[ \left[\begin{array}{rrr}
a & b & 5 \\
1 & -2 & 1
\end{array}\right] \rightarrow
\left[\begin{array}{rrr}
1 & -2 & 1 \\
a & b & 5
\end{array}\right] \rightarrow
\left[\begin{array}{ccc}
1 & -2 & 1 \\
0 & b+2a & 5-a
\end{array}\right]
\]

\pause
If $b+2a=0$ and $5-a=0$, i.e., $a=5$ and $b=-10$, then $\rank A=1$.

Otherwise, $\rank A=2$.
\end{solution}
}
%-------------- end slide -------------------------------%}}}
%-------------- start slide -------------------------------%{{{ 20
\frame{
\begin{block}{For {\bf any} system of linear equations, exactly one of the following holds:}\pause
    \begin{enumerate}
	\item the system is \alert{inconsistent}; \pause
	\item the system has a \alert{unique} solution, i.e., exactly one solution; \pause
	\item the system has \alert{infinitely many} solutions.
    \end{enumerate}
\end{block}
\vfill
\pause
\begin{emptytitle}
    One can see what case applies by looking at the RREF matrix equivalent to the augmented matrix of the system
    and distinguishing three cases:
    \begin{enumerate}
	\item The last nonzero row is \textcolor{teal}{$[0,\cdots, 0, 1]$}: no solution.
	\item The last nonzero row is \alert{not} \textcolor{teal}{$[0,\cdots, 0, 1]$} and all variables are leading: unique solution.
	\item The last nonzero row is \alert{not} \textcolor{teal}{$[0,\cdots, 0, 1]$} and there are non-leading variables:
	    infinitely many solutions.
    \end{enumerate}
\end{emptytitle}

}
%-------------- end slide -------------------------------%}}}
%-------------- start slide -------------------------------%{{{ 21
\frame{
\shiftup{0.5}
\begin{problem}
Solve the system
\[ \begin{array}{rrrrrrrrrrr}
    -3x_1 & + & 6x_2 & - & 4x_3 & - & 9x_4 & + & 3x_5 & = & -1 \\
    -x_1  & + & 2x_2 & - & 2x_3 & - & 4x_4 & - & 3x_5 & = & 3  \\
    x_1   & - & 2x_2 & + & 2x_3 & + & 2x_4 & - & 5x_5 & = & 1  \\
    x_1   & - & 2x_2 & + & x_3  & + & 3x_4 & - & x_5  & = & 1
\end{array}\]
\end{problem}
\shiftup{0.5}
\pause

\begin{solution}
Begin by putting the augmented matrix in reduced row-echelon form.
\[
\left[\begin{array}{rrrrr|r}
    1  & -2 & 2  & 2  & -5 & 1  \\
    -3 & 6  & -4 & -9 & 3  & -1 \\
    -1 & 2  & -2 & -4 & -3 & 3  \\
    1  & -2 & 1  & 3  & -1 & 1
\end{array}\right]
\pause
\rightarrow
\left[\begin{array}{rrrrr|r}
	    \alert<5>{1}   & -2                     & 0                      & 0                      & -13                    & 9 \\
    0                      & 0                      & \alert<5>{1}           & 0                      & 0                      & -2 \\
    0                      & 0                      & 0                      & \alert<5>{1}           & 4                      & -2 \\
    \textcolor<4>{teal}{0} & \textcolor<4>{teal}{0} & \textcolor<4>{teal}{0} & \textcolor<4>{teal}{0} & \textcolor<4>{teal}{0} & \textcolor<4>{teal}{0}
\end{array}\right]
\]
\pause
The system is \textcolor<4>{teal}{consistent}.
\pause
The rank of the augmented matrix is 3.
\pause

Since the system is consistent, the set of solutions has $5-3=2$ parameters.
\end{solution}
}
%-------------- end slide -------------------------------%}}}
%-------------- start slide -------------------------------%{{{ 22
\frame{
\begin{solution}[continued]
From the reduced row-echelon matrix
\[
\left[\begin{array}{rrrrr|r}
1 & -2 & 0 & 0 & -13 & 9 \\
0 & 0 & 1 & 0 & 0 & -2 \\
0 & 0 & 0 & 1 & 4 & -2 \\
0 & 0 & 0 & 0 & 0 & 0
\end{array}\right],
\]

\pause
we obtain the general solution
\[ \left.
\begin{array}{rrl}
x_1 & = & 9  +  2r  +  13s \\
x_2 & = & r \\
x_3 & = & -2 \\
x_4 & = & -2 - 4s\\
x_5 & = & s
\end{array} \right\} ~~\forall r,s\in\RR
\]

\pause{
The solution has two parameters ($r$ and $s$) as we expected.
}
\end{solution}
}
%-------------- end slide -------------------------------%}}}
\section[\textcolor{yellow}{}]{\textcolor{yellow}{Uniqueness of the Reduced Row-Echelon Form}}
%-------------- start slide -------------------------------%{{{ 23
\frame{
\frametitle{Uniqueness of the Reduced Row-Echelon Form}
\pause
\begin{theorem}
    Systems of linear equations that correspond to row equivalent
    augmented matrices have exactly the same solutions.
\end{theorem}
\vfill
\pause
\begin{theorem}
    Every matrix $A$ is row equivalent to a \alert{unique}
    reduced row-echelon matrix.
\end{theorem}
}
%-------------- end slide -------------------------------%}}}
%-------------- start slide -------------------------------%{{{ 24
\frame{
\shiftup{0.5}
\begin{problem}
Solve the system
\[ \begin{array}{ccccccc}
    2x & + & y & + & 3z & = & 1 \\
    2y & - & z & + & x  & = & 0 \\
    9z & + & x & - & 4y & = & 2
\end{array}\]
\end{problem}
\vfill
\pause
\begin{solution}
    \[
	\left[\begin{array}{rrr|r}
		2 & 1  & 3  & 1 \\
		1 & 2  & -1 & 0 \\
		1 & -4 & 9  & 2
	\end{array}\right]
	\rightarrow
	\left[\begin{array}{rrr|r}
		1 & 2  & -1 & 0 \\
		2 & 1  & 3  & 1 \\
		1 & -4 & 9  & 2
	\end{array}\right]
	\rightarrow
	\left[\begin{array}{rrr|r}
		1 & 2  & -1 & 0 \\
		0 & -3 & 5  & 1 \\
		0 & -6 & 10 & 2
	\end{array}\right]
    \]
    \shiftup{0.5}
    \[
	\rightarrow\left[\begin{array}{rrr|r}
		1 & 2  & -1 & 0 \\
		0 & -3 & 5  & 1 \\
		0 & 0  & 0  & 0
	\end{array}\right]
	\rightarrow \left[\begin{array}{rrr|r}
		1 & 2 & -1                          & 0                           \\
		0 & 1 & -\vspace{0.05in}\frac{5}{3} & -\vspace{0.05in}\frac{1}{3} \\
		0 & 0 & 0                           & 0
	\end{array}\right]
	\rightarrow \left[\begin{array}{rrr|r}
		1 & 0 & -\vspace{0.05in}\frac{7}{3} & -\vspace{0.05in}\frac{2}{3} \\
		0 & 1 & -\vspace{0.05in}\frac{5}{3} & -\vspace{0.05in}\frac{1}{3} \\
		0 & 0 & 0                           & 0
	\end{array}\right]
    \]
\end{solution}
}
%-------------- end slide -------------------------------%}}}
%-------------- start slide -------------------------------%{{{ 25
\frame{
\begin{solution}[continued]
    This row-echelon matrix corresponds to the system
    \[
    \begin{array}{ccccccc}
	x & + & 0y & + & \vspace{0.05in}\frac{7}{3}z & = & -\vspace{0.05in}\frac{2}{3} \\
	  &   & y  & - & \vspace{0.05in}\frac{5}{3}z & = & -\vspace{0.05in}\frac{1}{3}
    \end{array},
    \]
    \pause
    and thus
    \[
	\begin{array}{lcl}
	    x & = & \vspace{0.05in}\frac{2}{3}-\vspace{0.05in}\frac{7}{3}z\\
	    y & = & -\vspace{0.05in}\frac{1}{3} +\vspace{0.05in}\frac{5}{3}z
	\end{array}
    \]

    \pause
    Setting $z=s$, where $s\in\RR$, gives us (as before):
    \[
	\begin{array}{rrrrr}
	    x & = & \vspace{0.05in}\frac{2}{3} &  - & \vspace{0.05in}\frac{7}{3}s\\
	    y & = & -\vspace{0.05in}\frac{1}{3}&  + & \vspace{0.05in}\frac{5}{3}s\\
	    z & = & s & & \\
	\end{array}
    \]\pause
    {\bf Always check your answer!}
    \myQED
\end{solution}
}
%-------------- end slide -------------------------------%}}}
\section[\textcolor{yellow}{}]{\textcolor{yellow}{One Application}}
%-------------- start slide -------------------------------%{{{ 26
\begin{frame}[fragile]
\frametitle{One Application}
\pause
\begin{problem}
    Derive the formula for $1^r+2^r+\cdots + n^r$  for $r=3$.
\end{problem}
\pause
\vfill
\begin{solution}
    We know that $1^3 + 2^3 + \cdots + n^3$ is a polynomial in $n$ of oder $4$, namely,
    \begin{align*}
	1^3 + 2^3 + \cdots + n^3 = a_0+a_1 n + a_2 n^2 +a_3 n^3 + a_4 n^4.
    \end{align*}
    \pause
    It is easy to see that when $n=0$, both sides should be equal to zero. Hence, $a_0=0$.
    \pause
    Now we have $4$ unknowns, $a_1,\cdots, a_4$. We can let $n=1,\cdots,4$ to form $4$ equations in order to find these unknowns:
    \begin{align*}
	\begin{array}{ccccccccccc}
	    1^1a_1 & + & 1^2a_2 & + & 1^3a_3 & + & 1^4a_4 &  = & 1^3                 &  & \left(n=1\right) \\
	    2^1a_1 & + & 2^2a_2 & + & 2^3a_3 & + & 2^4a_4 &  = & 1^3+2^3             &  & \left(n=2\right) \\
	    3^1a_1 & + & 3^2a_2 & + & 3^3a_3 & + & 3^4a_4 &  = & 1^3+2^3+3^3         &  & \left(n=3\right) \\
	    4^1a_1 & + & 4^2a_2 & + & 4^3a_3 & + & 4^4a_4 &  = & 1^3+2^3+3^3+4^3     &  & \left(n=4\right) \\
	\end{array}
    \end{align*}
\end{solution}
\end{frame}
%-------------- end slide -------------------------------%}}}
%-------------- start slide -------------------------------%{{{ 27
\begin{frame}[fragile]
    \begin{solution}[continued]
    Hence, we have the following augmented matrix:
    \begin{align*}
	\left(
	    \begin{array}{ cccc|c }
	  1  &    1 &     1  &    1 &    1 \\
	  2  &    4 &     8  &   16 &   9\\
	  3  &    9 &    27  &   81 &   36 \\
	  4  &   16 &    64  &  256 &  100\\
	\end{array}
	\right)
    \end{align*}
    \pause
    You can use Octave or Matlab to compute the reduced echelon form:
    \begin{align*}
	\left(
	\begin{array}{ cccc|c }
	   1 & 0  & 0  & 0  &   0\\
	   0 & 1  & 0  & 0  &  1/4 \\
	   0 & 0  & 1  & 0  &  1/2 \\
	   0 & 0  & 0  & 1  &  1/4 \\
	\end{array}
	\right)
    \end{align*}
    \pause
    Therefore,  we have that
    \[
	1^3 + 2^3 + \cdots + n^3 = \frac{n^2}{4} + \frac{n^3}{2}+\frac{n^4}{4} = \frac{1}{4} n^2(n+1)^2.
    \]
    \myQED
\end{solution}
\end{frame}
%-------------- end slide -------------------------------%}}}
\end{document}

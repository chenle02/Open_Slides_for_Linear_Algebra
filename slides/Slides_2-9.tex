%%%%%%%%%%%%%%%%%%%%% {{{
%%Options for presentations (in-class) and handouts (e.g. print).
\documentclass[pdf,9pt]{beamer}


%%%%%%%%%%%%%%%%%%%%%%
%Change this for different slides so it appears in bar
\usepackage{authoraftertitle}
\date{Chapter 2. Matrix Algebra \\ \S 2-9. Markov Chains}

%%%%%%%%%%%%%%%%%%%%%%
%% Upload common style file
\usepackage{LyryxLAWASlidesStyle}

\begin{document}

%%%%%%%%%%%%%%%%%%%%%%%
%% Title Page and Copyright Common to All Slides

%Title Page
\input frontmatter/titlepage.tex

%LOTS Page
\input frontmatter/lyryxopentexts.tex

%Copyright Page
\input frontmatter/copyright.tex

%%%%%%%%%%%%%%%%%%%%%%%%% }}}
%-------------- start slide -------------------------------%{{{ 2
\begin{frame}[fragile]
   \tableofcontents
\end{frame}
%-------------- end slide -------------------------------%}}}
\section[\textcolor{yellow}{}]{\textcolor{yellow}{Markov Chains}}
%-------------- start slide -------------------------------%{{{ 3
\frame{
\frametitle{Markov Chains}
\pause
\begin{emptytitle}
{\bf Markov Chains} are used to model systems (or processes)
that evolve through a series of \alert{stages}.
At each stage, the system is in one of a finite number of
\alert{states}.
\end{emptytitle}
\vfill
\pause
\begin{example}[Weather Model]
    Three states: sunny (S), cloudy (C), rainy (R). \\
    Stages: days.\\
    The state that the system occupies at any stage is determined
    by a set of probabilities.
    \bigskip

    \alert{Important fact: probabilities are always real numbers
    between zero and one, inclusive.}
\end{example}
}
%-------------- end slide -------------------------------%}}}
%-------------- start slide -------------------------------%{{{ 4
\frame{
\begin{example}[Weather Model -- continued]
    \begin{itemize}
	\item If it is sunny one day, then there is a 40\% chance it will
	be sunny the next day, and a 40\% chance that it will be cloudy the
	next day \pause \alert{(and a 20\% chance it will be rainy
	the next day).}
    \end{itemize}
    \pause
    The values 40\%, 40\% and 20\% are
    \alert{transition probabilities}, and are assumed to be known.
    \begin{itemize}
	\item If it is cloudy one day, then there is a 40\% chance it will be rainy the next day, and a 25\% chance that it will be sunny the next day. \pause
	\item If it is rainy one day, then there is a 30\% chance it will be rainy the next day, and a 50\% chance that it will be cloudy the next day.
    \end{itemize}
\end{example}
}
%-------------- end slide -------------------------------%}}}
%-------------- start slide -------------------------------%{{{ 5
\frame{
\begin{example}[Weather Model -- continued]
    We put the transition probabilities into a
    \alert{transition matrix},
    \[ P=
	\left[\begin{array}{lll}
	    0.4 & 0.25 & 0.2 \\
	    0.4 & 0.35 & 0.5 \\
	    0.2 & 0.4  & 0.3
	\end{array}\right]
    \]

    {\bf Note.} Transition matrices are \alert{stochastic}, meaning that
    the sum of the entries in each column is equal to one.
    \bigskip

    \pause
    Suppose that it is rainy on Thursday.
    What is the probability that it will be sunny on Sunday?
    \bigskip

    \pause
    The \alert{initial state} vector, $S_0$, corresponds to the
    state of the weather on Thursday, so
    \[ S_0 =
    \left[\begin{array}{l}
	0 \\ 0 \\ 1
    \end{array}\right] \]
\end{example}
}
%-------------- end slide -------------------------------%}}}
%-------------- start slide -------------------------------%{{{ 6
\frame{
\begin{example}[Weather Model -- continued]
    What is the state vector for Friday?
    \pause
    \[
	S_1 =
	\left[\begin{array}{l}
	0.2 \\ 0.5 \\ 0.3
	\end{array}\right]
	=
	\left[\begin{array}{lll}
	    0.4 & 0.25 & 0.2 \\
	    0.4 & 0.35 & 0.5 \\
	    0.2 & 0.4  & 0.3
	\end{array}\right]
	\left[\begin{array}{l}
	0 \\ 0 \\ 1
	\end{array}\right]
	=
	PS_0.
    \]

    \pause
    To find the state vector for Saturday:
    \[
	S_2=PS_1=
	\left[\begin{array}{lll}
	    0.4 & 0.25 & 0.2 \\
	    0.4 & 0.35 & 0.5 \\
	    0.2 & 0.4  & 0.3
	\end{array}\right]
	\left[\begin{array}{l}
	0.2 \\ 0.5 \\ 0.3
	\end{array}\right]
	=
	\left[\begin{array}{l}
	0.265 \\ 0.405 \\ 0.33
	\end{array}\right]
    \]
    \pause
    Finally, the state vector for Sunday is
    \[
	S_3=PS_2=
	\left[\begin{array}{lll}
	    0.4 & 0.25 & 0.2 \\
	    0.4 & 0.35 & 0.5 \\
	    0.2 & 0.4  & 0.3
	\end{array}\right]
	\left[\begin{array}{l}
	0.265 \\ 0.405 \\ 0.33
	\end{array}\right]
	=
	\left[\begin{array}{l}
	0.27325 \\ 0.41275 \\ 0.314
	\end{array}\right]
    \]
    \pause
    The probability that it will be sunny on Sunday is 27.325\%.
    \bigskip

    \alert{Important fact: the sum of the entries of a state vector is always one.}
\end{example}
}
%-------------- end slide -------------------------------%}}}
%-------------- start slide -------------------------------%{{{ 7
\frame{
\begin{theorem}[\S2.9 Theorem 1]
    If $P$ is the transition matrix for an $n$-state Markov chain,
    then
    \[ S_{m+1}=PS_m \quad \mbox{ for } m=0, 1, 2, \ldots\]
\end{theorem}
\vfill
\pause
\begin{example}[\S2.9 Example 1]
    \begin{itemize}
	\item A customer always eats lunch either at restaurant $A$ or restaurant $B$.
	\item The customer never eats at $A$ two days in a row.
	\item If the customer eats at $B$ one day, then the next day she is three times as likely to eat at $B$ as at $A$.
    \end{itemize}
    What is the probability transition matrix?

    \pause
    \[ P =
	\left[\begin{array}{ll}
	    0 & 1/4 \\ 1 & 3/4
	\end{array}\right] \]
\end{example}
}
%-------------- end slide -------------------------------%}}}
%-------------- start slide -------------------------------%{{{ 8
\frame{
\begin{example}[continued]
    Initially, the customer is equally likely to eat at either restaurant,
    so
    \[ S_0 =
    \left[\begin{array}{l}
	1/2 \\ 1/2
    \end{array}\right] \]
    \[
	S_1 = \left[\begin{array}{l} 0.125     \\ 0.875     \end{array}\right],
	S_2 = \left[\begin{array}{l} 0.21875   \\ 0.78125   \end{array}\right],
	S_3 = \left[\begin{array}{l} 0.1953125 \\ 0.8046875 \end{array}\right],
    \]
    \[
	S_4 = \left[\begin{array}{l} 0.20117 \\ 0.79883 \end{array}\right],
	S_5 = \left[\begin{array}{l} 0.19971 \\ 0.80029 \end{array}\right],
    \]
    \[
	S_6 = \left[\begin{array}{l} 0.20007 \\ 0.79993 \end{array}\right],
	S_7 = \left[\begin{array}{l} 0.19998 \\ 0.80002 \end{array}\right],
    \]
    are calculated, and these
    appear to converge to
    \[ \left[\begin{array}{l}
	0.2 \\ 0.8
    \end{array}\right] \]
\end{example}
}
%-------------- end slide -------------------------------%}}}
%-------------- start slide -------------------------------%{{{ 9
\frame{
\begin{example}[\S2.9 Example 3]
    A wolf pack always hunts in one of three regions, $R_1$, $R_2$, and $R_3$.
    \begin{itemize}
	\item If it hunts in some region one day, it is as likely as not to hunt there again the next day.
	\item If it hunts in $R_1$, it never hunts in $R_2$ the next day.
	\item If it hunts in $R_2$ or $R_3$, it is equally likely to hunt in each of the other two regions the next day.
    \end{itemize}
    If the pack hunts in $R_1$ on Monday, find the probability that
    it will hunt in \alert{$R_3$ on Friday}.

    \pause
    \[ P =
    \left[\begin{array}{lll}
	1/2 & 1/4 & 1/4 \\
	0 & 1/2 & 1/4 \\
	1/2 & 1/4 & 1/2
    \end{array}\right]
    \quad\text{and}\quad
    S_0 =
    \left[\begin{array}{l}
    1 \\ 0 \\ 0
    \end{array}\right]
    \]
    \pause
    We want to find $S_4$, and, in particular, the last entry in $S_4$.
\end{example}
}
%-------------- end slide -------------------------------%}}}
%-------------- start slide -------------------------------%{{{ 10
\frame{
\begin{example}[continued]
    \[
	    S_1 =
	\left[\begin{array}{l}
	    1/2 \\
	    0 \\
	    1/2
	\end{array}\right],
    \]
    \[
	S_2 = PS_1=
	\left[\begin{array}{lll}
	    1/2 & 1/4 & 1/4 \\
	    0   & 1/2 & 1/4 \\
	    1/2 & 1/4 & 1/2
	\end{array}\right]
	\left[\begin{array}{l}
	    1/2 \\
	    0 \\
	    1/2
	\end{array}\right]
	=
	\left[\begin{array}{l}
	    3/8 \\
	    1/8 \\
	    1/2
	\end{array}\right],
    \]
    \pause
    \[
	S_3 = PS_2=
	\left[\begin{array}{lll}
	    1/2 & 1/4 & 1/4 \\
	    0   & 1/2 & 1/4 \\
	    1/2 & 1/4 & 1/2
	\end{array}\right]
	\left[\begin{array}{l}
	    3/8 \\
	    1/8 \\
	    1/2
	\end{array}\right]
	=
	\left[\begin{array}{l}
	    11/32 \\
	    3/16 \\
	    15/32
	\end{array}\right],
    \]
    \pause
    \[
	S_4 = PS_3=
	\left[\begin{array}{lll}
	    1/2 & 1/4 & 1/4 \\
	    0   & 1/2 & 1/4 \\
	    1/2 & 1/4 & 1/2
	\end{array}\right]
	\left[\begin{array}{l}
	    11/32 \\
	    3/16 \\
	    15/32
	\end{array}\right]
	=
	\left[\begin{array}{l}
	     \star \\
	     \star \\
	    29/64
	\end{array}\right].
    \]
   \pause
   \bigskip

   Therefore, the probability of the pack hunting in $R_3$ on Friday is $29/64$.
   \myQED
\end{example}
}
%-------------- end slide -------------------------------%}}}
%-------------- start slide -------------------------------%{{{ 11
\frame{
\begin{emptytitle}
    Sometimes, state vectors converge to a particular vector,
    called the \alert{steady state} vector.
\end{emptytitle}
\pause
\begin{problem}
    How do we know if a Markov chain has a steady state vector?
    If the Markov chain has a steady state vector, how do we
    find it?
\end{problem}
\vfill
\pause
\begin{emptytitle}
    One condition ensuring that a steady state vector
    exists is that the transition
    matrix $P$ be \alert{regular}, meaning that
    for some integer $k>0$, all entries of $P^k$ are
    \alert{positive} (i.e., greater than zero).
\end{emptytitle}
\pause
\vfill
\begin{example}
    In \S2.9 Example 1,
    $ P =
    \left[\begin{array}{ll}
    0 & 1/4 \\ 1 & 3/4
    \end{array}\right]$ \alert{is regular} because
    \[P^2 =
    \left[\begin{array}{ll}
    0 & 1/4 \\ 1 & 3/4
    \end{array}\right]
    \left[\begin{array}{ll}
    0 & 1/4 \\ 1 & 3/4
    \end{array}\right]
    =
    \left[\begin{array}{ll}
    1/4 & 3/16 \\ 3/4 & 3/16
    \end{array}\right]
    \]
    has all entries greater than zero.
\end{example}
}
%-------------- end slide -------------------------------%}}}
%-------------- start slide -------------------------------%{{{ 12
\frame{
\begin{theorem}[\S2.9 Theorem 2 -- paraphrased]
    If $P$ is the transition matrix of a Markov chain and $P$
    is regular, then the steady state vector can be found by
    solving the system
    \[ S=PS \]
    for $S$, and then ensuring that the entries of $S$ sum to one.
\end{theorem}
\vfill
\pause
\begin{emptytitle}
    Notice that if $S=PS$, then
    \begin{eqnarray*}
	S-PS   & = & 0 \\
	IS-PS  & = & 0 \\
	(I-P)S & = & 0
    \end{eqnarray*}
    \begin{itemize}
	\item This last line represents a system of linear equations that is homogeneous. \pause
	\item The structure of $P$ ensures that $I-P$ is not invertible, and so the system has infinitely many solutions.\pause
	\item Choose the value of the parameter so that the entries of $S$ sum to one.
    \end{itemize}
\end{emptytitle}
}
%-------------- end slide -------------------------------%}}}
%-------------- start slide -------------------------------%{{{ 13
\frame{
\begin{example}
    From \S2.9 Example 1,
    \[
	P=
	\left[\begin{array}{ll}
	0 & 1/4 \\
	1 & 3/4
	\end{array}\right],\]
    and we've already verified that $P$ is regular.

    \medskip
    \pause
    Now solve the system $(I-P)S=0$.
    \pause
    \[ I-P =
    \left[\begin{array}{rr}
	1 & 0 \\
	0 & 1
    \end{array}\right]
    -
    \left[\begin{array}{rr}
	0 & 1/4  \\
	1 & 3/4
    \end{array}\right]
    =
    \left[\begin{array}{rr}
	1  & -1/4 \\
	-1 & 1/4
    \end{array}\right]
    \]
    \pause
    Solving $(I-P)S=0$:
    \[
    \left[\begin{array}{rr|r}
	1  & -1/4 & 0 \\
	-1 & 1/4  & 0
    \end{array}\right]
    \rightarrow
    \left[\begin{array}{rr|r}
	1 & -1/4 & 0 \\
	0 & 0    & 0
    \end{array}\right]
    \]
    \pause
    The general solution in parametric form is
    \[ s_1 = \frac{1}{4}t,\quad s_2=t \mbox{ for } t\in\RR.\]
\end{example}
}
%-------------- end slide -------------------------------%}}}
%-------------- start slide -------------------------------%{{{ 14
\frame{
\begin{example}[continued]
    Since $s_1+s_2=1$,
    \begin{eqnarray*}
	\frac{1}{4} t + t & = & 1 \\
	\frac{5}{4} t     & = & 1 \\
	t                 & = & \frac{4}{5}
    \end{eqnarray*}
    \pause
    Therefore, the steady state vector is
    \[
	S=\left[\begin{array}{l} 1/5 \\ 4/5 \end{array}\right]
	=\left[\begin{array}{l}  0.2         \\ 0.8         \end{array}\right]
    \]
    \myQED
\end{example}
}
%-------------- end slide -------------------------------%}}}
%-------------- start slide -------------------------------%{{{ 15
\frame{
\begin{example}[\S2.9 Example 3]
    Is there a steady state vector?
    If so, find it.
    \pause
    \[ P =
    \left[\begin{array}{lll}
	1/2 & 1/4 & 1/4 \\
	0   & 1/2 & 1/4 \\
	1/2 & 1/4 & 1/2
    \end{array}\right] \]

    \pause
    so
    \[ P^2=
    \left[\begin{array}{lll}
	1/2 & 1/4 & 1/4 \\
	0   & 1/2 & 1/4 \\
	1/2 & 1/4 & 1/2
    \end{array}\right]
    \left[\begin{array}{lll}
	1/2 & 1/4 & 1/4 \\
	0   & 1/2 & 1/4 \\
	1/2 & 1/4 & 1/2
    \end{array}\right]
    =
    \left[\begin{array}{lll}
	5/8 & 5/16 & 5/16 \\
	1/8 & 5/16 & 1/4  \\
	1/2 & 3/8  & 7/16
    \end{array}\right]
    \]
    \bigskip

    \pause
    Therefore $P$ is regular.
\end{example}
}
%-------------- end slide -------------------------------%}}}
%-------------- start slide -------------------------------%{{{ 16
\frame{
\begin{example}[continued]
    Now solve the system $(I-P)S=0$.
    \[
    \left[\begin{array}{rrr|r}
	1/2  & -1/4 & -1/4 & 0 \\
	0    & 1/2  & -1/4 & 0 \\
	-1/2 & -1/4 & 1/2  & 0
    \end{array}\right]
    \rightarrow
    \left[\begin{array}{rrr|r}
	1/2 & -1/4 & -1/4 & 0 \\
	0   & 1/2  & -1/4 & 0 \\
	0   & -1/2 & 1/4  & 0
	\end{array}\right]
    \]
    \[
    \rightarrow
    \left[\begin{array}{rrr|r}
	1 & -1/2 & -1/2 & 0 \\
	0 & 1/2  & -1/4 & 0 \\
	0 & 0    & 0    & 0
    \end{array}\right]
    \rightarrow
    \left[\begin{array}{rrr|r}
	1 & 0   & -3/4 & 0 \\
	0 & 1/2 & -1/4 & 0 \\
	0 & 0   & 0    & 0
    \end{array}\right]
    \]
    \[
    \rightarrow
    \left[\begin{array}{rrr|r}
	1 & 0 & -3/4 & 0 \\
	0 & 1 & -1/2 & 0 \\
	0 & 0 & 0    & 0
    \end{array}\right]
    \]
    \pause
    The general solution in parametric form is
    \[ s_3=t, \quad s_2=\frac{1}{2}t, \quad s_1=\frac{3}{4}t, \mbox{ where } t\in\RR.\]
\end{example}
}
%-------------- end slide -------------------------------%}}}
%-------------- start slide -------------------------------%{{{ 17
\frame{
\begin{example}[continued]
    Since $s_1 + s_2 + s_3 = 1$,
    \[ t +\frac{1}{2}t + \frac{3}{4}t = 1,\]
    implying that $t=\frac{4}{9}$.
    Therefore, the steady state vector is
    \[ S=\left[\begin{array}{r}
	3/9 \\
	2/9 \\
	5/9
    \end{array}\right].\]
    \myQED
\end{example}
}

%-------------- end slide -------------------------------%}}}
\end{document}

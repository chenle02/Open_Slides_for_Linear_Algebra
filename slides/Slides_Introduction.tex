%%%%%%%%%%%%%%%%%%%%% {{{
%%Options for presentations (in-class) and handouts (e.g. print).
% \documentclass[pdf,handout,9pt]{beamer}
\documentclass[pdf,9pt]{beamer}


%%%%%%%%%%%%%%%%%%%%%%
%Change this for different slides so it appears in bar
\usepackage{authoraftertitle}
\date{Introduction}

%%%%%%%%%%%%%%%%%%%%%%
%% Upload common style file
\usepackage{LyryxLAWASlidesStyle}

\begin{document}

%%%%%%%%%%%%%%%%%%%%%%%
%% Title Page and Copyright Common to All Slides

%Title Page
\input frontmatter/titlepage.tex

%LOTS Page
% \input frontmatter/lyryxopentexts.tex

%Copyright Page
% \input frontmatter/copyright.tex

%%%%%%%%%%%%%%%%%%%%%%%%% }}}
%-------------- start slide -------------------------------%{{{


%-------------- start slide -------------------------------%{{{
\frame{\frametitle{Motivation}
\begin{example}
    Find all solutions of the (linear) equation in one variable:
    \[
	ax=b
    \]
\end{example}
\pause
\vfill
\begin{solution}
    \begin{itemize}
	\item If $a \neq 0$, there is  a unique solution $x=b/a$.
	\pause
	\item Else if $a = 0$ and
	\item[] $b\neq 0$, there is  no solution.
	 \pause
     \item[] $b =  0$, there are infinitely many solutions, in fact any $x \in \RR$ is a solution.
    \end{itemize}
    This a complete description of all possible solutions of $ax=b$.
\end{solution}
\pause
\begin{alertblock}{Objective:}
Can we do the same for linear equations in more variables?
\end{alertblock}
}
%-------------- end slide -------------------------------%}}}

%-------------- start slide -------------------------------%{{{
\frame{
    \[
A = LU =
    \]
 \begin{tikzpicture}[
        %Global config
        >=latex,
        line width=1pt,
        %Styles
        Brace/.style={
            decorate,
            decoration={
                brace,
                raise=-7pt
            }
        }
    ]

    \matrix[% General option for all nodes
        matrix of nodes,
        text height=2.5ex,
        text depth=0.75ex,
        text width=3.25ex,
        align=center,
        left delimiter=(,
        right delimiter= ),
        column sep=5pt,
        row sep=5pt,
        %nodes={draw=black!10}, % Uncoment to see the square nodes.
        nodes in empty cells,
    ] at (0,0) (M){ % Matrix contents
    &   \huge 0&   &   &   &   &  \\
    &   &   &   &   &   &  \\
    &   &   &   &   &   &  \\
    &   &   &   &   &   &  \\
    &   &   &   &   &   &  \\
    &   &   &   &   &   & \huge 0 \\
    };
% Drawing the sectors using matrix coordinate names.
    \draw[thick] (M-1-2) -- (M-6-7);
    \draw[thick,fill=red!30,draw] (M-1-3.center)
    -- (M-4-6.center)
    -- (M-1-6.center)
    -- cycle;
    \draw[thick,fill=red!30,draw] (M-2-2.center)
    -- (M-5-2.center)
    -- (M-5-5.center)
    -- cycle;
    \draw[thick,fill=yellow!30,draw](M-1-6.east)
    -- (M-1-7.center)
    -- (M-5-7.center)
    -- (M-4-6.south east)
    -- cycle;
    \draw[thick,fill=yellow!30,draw](M-5-2.south)
    -- (M-5-5.south east)
    -- (M-6-6.center)
    -- (M-6-2.center)
    -- cycle;
% Drawing the braces.
    \draw[Brace] (M-1-3.north)
    -- (M-1-6.north)
    node[midway,above]{$x-y$};
    \draw[Brace] (M-1-6.north east)
    -- (M-1-7.north)
    node[midway,above]{$y$};

    \draw[Brace] (M-5-2.west)
    -- (M-2-2.west)
    node[midway,left]{$x-y$};

    \draw[Brace] (M-6-2.west)
    -- (M-5-2.south west)
    node[midway,left]{$y$};
% Labeling the sectors
    \node at (M-4-3){\sf A};
    \node at (M-2-5){\sf A};
    \draw (M-3-7)++(-8pt,0) node {\sf B};
    \draw (M-6-4)++(0,8pt) node {\sf B};

    \end{tikzpicture}
}
%-------------- end slide -------------------------------%}}}

%-------------- start slide -------------------------------%{{{
%-------------- end slide -------------------------------%}}}
\end{document}

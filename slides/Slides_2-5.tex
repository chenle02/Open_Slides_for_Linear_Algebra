%%%%%%%%%%%%%%%%%%%%% {{{
%%Options for presentations (in-class) and handouts (e.g. print).
\documentclass[pdf,9pt]{beamer}
% \documentclass[pdf,9pt]{beamer}


%%%%%%%%%%%%%%%%%%%%%%
%Change this for different slides so it appears in bar
\usepackage{authoraftertitle}
\date{Chapter 2. Matrix Algebra \\ \S 2-5. Elementary Matrices}

%%%%%%%%%%%%%%%%%%%%%%
%% Upload common style file
\usepackage{LyryxLAWASlidesStyle}

\begin{document}

%%%%%%%%%%%%%%%%%%%%%%%
%% Title Page and Copyright Common to All Slides

%Title Page
\input frontmatter/titlepage.tex

%LOTS Page
\input frontmatter/lyryxopentexts.tex

%Copyright Page
\input frontmatter/copyright.tex

%%%%%%%%%%%%%%%%%%%%%%%%% }}}
%-------------- start slide -------------------------------%{{{ 2
\begin{frame}[fragile]
   \tableofcontents
\end{frame}
%-------------- end slide -------------------------------%}}}
\section[\textcolor{yellow}{}]{\textcolor{yellow}{Elementary Matrices}}
%-------------- start slide -------------------------------%{{{ 3
\frame{
\frametitle{Elementary Matrices}
\pause
\begin{definition}
    An \alert{elementary matrix} is a matrix obtained from an
    identity matrix by performing \alert{a single} elementary
    row operation.
    % \pause
    % The \alert{type} of an elementary matrix is given by the type of row operation
    % used to obtain the elementary matrix.
\end{definition}
\pause
\vfill
\begin{remark}[ Three Types of Elementary Row Operations ]
    ($\sim$ bases for genomic sequences)
\begin{itemize}
    \item {\bf Type I:} Interchange two rows.
    \item {\bf Type II:} Multiply a row by a nonzero number.
    \item {\bf Type III:}
    Add a (nonzero) multiple of one row to a different row.
\end{itemize}
\end{remark}
}
%-------------- end slide -------------------------------%}}}
%-------------- start slide -------------------------------%{{{ 4
\frame{
\begin{example}
    \begin{minipage}{0.3\textwidth}
    \begin{center}
       Switch the 2nd row and the 4th row
    \end{center}
    \end{minipage}
    \hfill
    \begin{minipage}{0.3\textwidth}
    \begin{center}
       Multiply $-2$ to the 3rd row
    \end{center}
    \end{minipage}
    \hfill
    \begin{minipage}{0.3\textwidth}
    \begin{center}
       Add $-3$ multiple of 1st row to the 3rd row
    \end{center}
    \end{minipage}
    \bigskip
    \[ E =
    \left[ \begin{array}{cccc}
	1 & 0 & 0 & 0 \\
	0 & 0 & 0 & 1 \\
	0 & 0 & 1 & 0 \\
	0 & 1 & 0 & 0
    \end{array}\right],
    F=
    \left[ \begin{array}{rrrr}
	1 & 0 & 0  & 0 \\
	0 & 1 & 0  & 0 \\
	0 & 0 & -2 & 0 \\
	0 & 0 & 0  & 1
    \end{array}\right],
    G=
    \left[ \begin{array}{rrrr}
	1  & 0 & 0 & 0 \\
	0  & 1 & 0 & 0 \\
	-3 & 0 & 1 & 0 \\
	0  & 0 & 0 & 1
    \end{array}\right],
    \]
    \bigskip
    are examples of elementary matrices of types I, II and III, respectively.
\end{example}
}
%-------------- end slide -------------------------------%}}}
%-------------- start slide -------------------------------%{{{ 4
\frame{
    \begin{example}[continued]
    Let
    \[A= \left[ \begin{array}{rr}
    1 & 1 \\ 2 & 2 \\ 3 & 3 \\ 4 & 4
    \end{array}\right] \]
    \pause
    We are interested in the effect that (left) multiplication of
    $A$ by $E$, $F$ and $G$ has on the matrix $A$.
    \pause
    Computing $EA$, $FA$, and $GA$ \ldots
\end{example}
}
%-------------- end slide -------------------------------%}}}
%-------------- start slide -------------------------------%{{{ 5
\frame{
\begin{example}[continued]


    \begin{minipage}{0.65\textwidth}
    \[
	EA=
	\left[ \begin{array}{cccc}
		1 & 0                  & 0 & 0                   \\
		0 & 0                  & 0 & \textcolor{blue}{1} \\
		0 & 0                  & 1 & 0                   \\
		0 & \textcolor{red}{1} & 0 & 0
	\end{array}\right]
	\left[ \begin{array}{rr}
		1 & 1 \\ \alert{2} & \alert{2} \\ 3 & 3 \\ \textcolor{blue}{4} &\textcolor{blue}{4}
	\end{array}\right]=
	\left[ \begin{array}{rr}
		1 & 1 \\
		\textcolor{blue}{4} & \textcolor{blue}{4} \\
		3 & 3 \\
		\alert{2} &\alert{2}
	\end{array}\right]
    \]
    \end{minipage}
    \hfill
    \begin{minipage}{0.3\textwidth}
    \begin{center}
       Switch the 2nd row and the 4th row
    \end{center}
    \end{minipage}

    \pause
    \begin{minipage}{0.65\textwidth}
    \[
	FA=
	\left[ \begin{array}{rrrr}
		1 & 0 & 0                   & 0 \\
		0 & 1 & 0                   & 0 \\
		0 & 0 & \textcolor{red}{-2} & 0 \\
		0 & 0 & 0                   & 1
	\end{array}\right]
	\left[ \begin{array}{rr}
		1 & 1 \\ 2 & 2 \\ \textcolor{red}{3} & \textcolor{red}{3} \\ 4 & 4
	\end{array}\right]=
	\left[ \begin{array}{rr}
		1 & 1 \\
		2 & 2 \\
		\textcolor{red}{-6} & \textcolor{red}{-6} \\
		4 & 4 \\
	\end{array}\right]
    \]
    \end{minipage}
    \hfill
    \begin{minipage}{0.3\textwidth}
    \begin{center}
       Multiply $-2$ to the 3rd row
    \end{center}
    \end{minipage}
    \pause

    \begin{minipage}{0.65\textwidth}
    \[
	GA=
	\left[ \begin{array}{rrrr}
		\textcolor{yellow}{1} & 0 & 0         & 0 \\
		0                     & 1 & 0         & 0 \\
		\alert{-3}            & 0 & \alert{1} & 0 \\
		0                     & 0 & 0         & 1
	\end{array}\right]
	\left[ \begin{array}{rr}
		\textcolor{yellow}{1} & \textcolor{yellow}{1} \\ 2 & 2 \\ \alert{3} & \alert{3} \\ 4 & 4
	\end{array}\right]=
	\left[ \begin{array}{rr}
		\textcolor{yellow}{1} & \textcolor{yellow}{1} \\
		2                     & 2                     \\
		\alert{0}             & \alert{0}             \\
		4                     & 4                     \\
	\end{array}\right]
    \]
    \end{minipage}
    \hfill
    \begin{minipage}{0.3\textwidth}
    \begin{center}
       Add $-3$ multiple of 1st row to the 3rd row
    \end{center}
    \end{minipage}

    \myQED
\end{example}
\vfill
\begin{remark}
    The elementary matrices are the programmed receipts for your cooking!
\end{remark}
}
%-------------- end slide -------------------------------%}}}
%-------------- start slide -------------------------------%{{{ 6
\frame{
\begin{theorem}[Multiplication by an Elementary Matrix]
    Let $A$ be an $m\times n$ matrix.
    \begin{itemize}
	\item[If]  $B$ is obtained from $A$ by performing \alert{one single
	    elementary} row operation,
	\item[then] $B=EA$
	\item[] where $E$ is the elementary matrix obtained from $I_m$ by performing the same elementary operation on $I_m$ as was performed on $A$.
    \end{itemize}
\end{theorem}
\vfill
\begin{center}
    \def\x{0.5}
    \begin{tikzpicture}[scale=1, transform shape]
    \tikzset{>=latex}
    \node (A) at (0,\x) {$A$};
    \node (B) at (1,\x) {$B$};
    \node (I) at (0,-\x) {$I$};
    \node (E) at (1,-\x) {$E$};
    \draw[->] (A) -- (B);
    \draw[->] (I) -- (E);
    \node (O) at (\x,0) {\footnotesize El. Op.};
    \node (=) at (4.5*\x,0) {$\Longrightarrow$};

    \node (AEB) at (8*\x,0) {$A=EB$};

    % \coordinate (P) at (0.2,1.5);
    % \coordinate (Q) at (3,0.3);
    % \coordinate (0) at (0,0);
    % \draw (P) -- coordinate[pos={1/3}] (A) coordinate[pos={2/3}] (B) (Q) node [right] {$Q$};
    % \draw [->,dashed] (0) node [below] {$} -- (P) node [above] {$P$};
    % \draw [->,dashed] (0) -- (A) node [above] {$A$};
    % \draw [->,dashed] (0) -- (B) node [above] {$B$};
    % \filldraw (P) circle (0.05);
    % \filldraw (Q) circle (0.05);
    % \filldraw (0) circle (0.05);
    % \filldraw (A) circle (0.05);
    % \filldraw (B) circle (0.05);
    \end{tikzpicture}
\end{center}
}
%-------------- end slide -------------------------------%}}}
%-------------- start slide -------------------------------%{{{ 7
\frame{
\begin{problem}
    Let
    % \vspace*{-.3in}

    \[ A=\left[\begin{array}{rr}
    4 & 1 \\ 1 & 3 \end{array}\right]
    \quad\text{and}\quad
    C=\left[\begin{array}{rr}
    1 & 3 \\ 2 & -5 \end{array}\right] \]
    Find elementary matrices $E$ and $F$ so that
    $C=FEA$.
\end{problem}
\pause
\vfill
\begin{solution}
    {\bf Note.}
    The statement of the problem implies that \alert{$C$ can be
	obtained from $A$ by a sequence of two elementary row
    operations}, represented by elementary matrices $E$ and $F$.
    \pause
    \[
	A = \left[\begin{array}{rr}
		4 & 1 \\ 1 & 3
	\end{array}\right]
	\pause
	\stackrel{\rightarrow}{_E}
	\left[\begin{array}{rr}
		1 & 3 \\
		4 & 1
	\end{array}\right]
	\pause
	\stackrel{\rightarrow}{_F}
	\left[\begin{array}{rr}
		1 & 3 \\
		2 & -5
	\end{array}\right] = C
    \]
    \pause
    where $E=\left[\begin{array}{rr} 0 & 1 \\ 1  & 0 \end{array}\right]$ \pause
    and   $F=\left[\begin{array}{rr} 1 & 0 \\ -2 & 1 \end{array}\right]$.\pause
    Thus we have the sequence
    $A\rightarrow EA \rightarrow F(EA)=C$, \pause
    so $C=FEA$, i.e.,
    \[
	\left[\begin{array}{rr} 1 & 3 \\ 2  & -5 \end{array}\right] =
	\left[\begin{array}{rr} 1 & 0 \\ -2 & 1 \end{array}\right]
	\left[\begin{array}{rr} 0 & 1 \\ 1  & 0 \end{array}\right]
	\left[\begin{array}{rr} 4 & 1 \\ 1  & 3 \end{array}\right].
    \]
    \myQED
\end{solution}
}
%-------------- end slide -------------------------------%}}}
\section[\textcolor{yellow}{}]{\textcolor{yellow}{Inverses of elementary matrices}}
%-------------- start slide -------------------------------%{{{ 8
\frame{
\frametitle{Inverses of Elementary Matrices}
\pause
\begin{lemma}
    Every elementary matrix $E$ is invertible, and $E^{-1}$ is also an elementary matrix (of the same type).
    Moreover, $E^{-1}$ corresponds to the inverse of the row operation that produces $E$.
\end{lemma}
\vfill
\pause
\begin{emptytitle}
    The following table gives the inverse of each type of elementary row operation:
    \begin{table}[H]
	\centering
	\begin{tabu}{|c|c|c|}
	\hline
	\textbf{Type} & \textbf{Operation} & \textbf{Inverse Operation} \\ \hline
	I & Interchange rows $p$ and $q$ & Interchange rows $p$ and $q$ \\
	II & Multiply row $p$ by $k \neq 0$ & Multiply row $p$ by $1/k$ \\
	III & Add $k$ times row $p$ to row $q \neq p$ & Subtract $k$ times row $p$ from row $q$ \\ \hline
	\end{tabu}
    \end{table}
    \noindent Note that elementary matrices of type I are self-inverse.
\end{emptytitle}

}
%-------------- end slide -------------------------------%}}}
%-------------- start slide -------------------------------%{{{ 9
\frame{\frametitle{Inverses of Elementary Matrices}
\begin{example}
    Without using the matrix inversion algorithm, find the
    inverse of the elementary matrix
    \[ G=
    \left[ \begin{array}{rrrr}
	1  & 0 & 0 & 0 \\
	0  & 1 & 0 & 0 \\
	-3 & 0 & 1 & 0 \\
	0  & 0 & 0 & 1
    \end{array}\right]
    \]
    \pause
    {\bf Hint.}
    What row operation can be applied to $G$ to transform it
    to $I_4$?
    \pause
    The row operation $G\rightarrow I_4$ is to \alert{add} three times
    row one to row three,
    \pause
    and thus
    \[ G^{-1} =
    \left[ \begin{array}{cccc}
	1 & 0 & 0 & 0 \\
	0 & 1 & 0 & 0 \\
	3 & 0 & 1 & 0 \\
	0 & 0 & 0 & 1
    \end{array}\right] \]
    \pause
    \alert{Check by computing $G^{-1}G$.}
\end{example}
}
%-------------- end slide -------------------------------%}}}
%-------------- start slide -------------------------------%{{{ 10
\frame{
\begin{example}[continued]
    Similarly,
    \[ E^{-1} =
    \left[ \begin{array}{cccc}
	1 & 0 & 0 & 0 \\
	0 & 0 & 0 & 1 \\
	0 & 0 & 1 & 0 \\
	0 & 1 & 0 & 0
    \end{array}\right]^{-1} =
    \left[ \begin{array}{cccc}
	1 & 0 & 0 & 0 \\
	0 & 0 & 0 & 1 \\
	0 & 0 & 1 & 0 \\
	0 & 1 & 0 & 0
    \end{array}\right]
    \]
    \pause
    and
    \[ F^{-1} =
    \left[ \begin{array}{rrrr}
	1 & 0 & 0  & 0 \\
	0 & 1 & 0  & 0 \\
	0 & 0 & -2 & 0 \\
	0 & 0 & 0  & 1
    \end{array}\right]^{-1}
    = \left[ \begin{array}{rrrr}
	1 & 0 & 0            & 0 \\
	0 & 1 & 0            & 0 \\
	0 & 0 & -\frac{1}{2} & 0 \\
	0 & 0 & 0            & 1
    \end{array}\right]
    \]
\end{example}
}
%-------------- end slide -------------------------------%}}}
%-------------- start slide -------------------------------%{{{ 11
\frame{
\begin{emptytitle}
    Suppose $A$ is an $m\times n$ matrix and that $B$ can be
    obtained from $A$ by a sequence of $k$ elementary row operations.
    \pause
    Then there exist elementary matrices $E_1, E_2,\ldots E_k$
    such that
    \[ B=E_k(E_{k-1}(\cdots(E_2(E_1A))\cdots)) \]
    \pause
    Since matrix multiplication is associative, we have
    \[ B=(E_k E_{k-1} \cdots E_2 E_1)A\]
    \pause
    or, more concisely, $B=UA$ where
    $U=E_k E_{k-1} \cdots E_2 E_1$.
    \pause
    \bigskip

    To find $U$ so that $B=UA$, we \alert{could} find
    $E_1, E_2, \ldots, E_k$ and multiply these together
    (in the correct order), but there is an easier method for
    finding $U$.
\end{emptytitle}
}
%-------------- end slide -------------------------------%}}}
%-------------- start slide -------------------------------%{{{ 12
\frame{
\begin{definition}
    Let $A$ be an $m\times n$ matrix.  We write
    \[ A\rightarrow B \]
    if $B$ can be obtained from $A$ by a sequence of elementary
    row operations.
    \pause
    In this case, we call $A$ and $B$ are \alert{row-equivalent}.
\end{definition}
\pause
\begin{theorem}
    Suppose $A$ is an $m\times n$ matrix and that $A\rightarrow B$.
    Then
    \pause
    \begin{enumerate}
	\item there exists an  \alert{invertible} $m\times m$ matrix $U$ such that
	    $B=UA$;
	    \pause
	\item $U$ can be computed by performing elementary row operations on
	    $\left[\begin{array}{c|c} A & I_m \end{array}\right]$
	    to transform it into
	    $\left[\begin{array}{c|c} B & U \end{array}\right]$;
	    \pause
	\item $U=E_k E_{k-1} \cdots E_2 E_1$, where
	    $E_1, E_2, \ldots, E_k$ are elementary matrices
	    corresponding, in order,
	    to the elementary row operations
	    used to obtain $B$ from $A$.
    \end{enumerate}
\end{theorem}
}
%-------------- end slide -------------------------------%}}}
%-------------- start slide -------------------------------%{{{ 13
\frame{
\begin{problem}
    Let $A=\left[\begin{array}{rrr}
    3 & 0 & 1 \\ 2 & -1 & 0 \end{array}\right]$,
    and let $R$ be the reduced row-echelon form of $A$.

    Find a matrix $U$ so that $R=UA$.
\end{problem}
%%Also, find a matrix $Q$ so that $A=QR$.
\pause
\vfill
\begin{solution}
    \[
    \left[\begin{array}{rrr|rr}
	3 & 0  & 1 & 1 & 0 \\
	2 & -1 & 0 & 0 & 1
    \end{array}\right]
    \rightarrow
    \left[\begin{array}{rrr|rr}
	1 & 1  & 1 & 1 & -1 \\
	2 & -1 & 0 & 0 & 1
    \end{array}\right]
    \rightarrow
    \left[\begin{array}{rrr|rr}
	1 & 1  & 1  & 1  & -1 \\
	0 & -3 & -2 & -2 & 3
    \end{array}\right]
    \]
    \[
    \rightarrow
    \left[\begin{array}{rrr|rr}
	1 & 1 & 1   & 1   & -1 \\
	0 & 1 & 2/3 & 2/3 & -1
    \end{array}\right]
    \rightarrow
    \left[\begin{array}{rrr|rr}\vspace*{.02in}
	1 & 0 & 1/3 & 1/3 & 0 \\
	0 & 1 & 2/3 & 2/3 & -1
    \end{array}\right]
    \]
    \pause
    \bigskip

    Starting with
    $\left[\begin{array}{c|c}
    A & I \end{array}\right]$, we've obtained
    $\left[\begin{array}{c|c}
    R & U \end{array}\right]$.
    \pause
    \bigskip

    Therefore $R=UA$, where
    \[ U=\left[\begin{array}{rr}\vspace*{.02in}
	1/3 & 0 \\
	2/3 & -1
    \end{array}\right].
    \]
    \myQED
\end{solution}
}
%-------------- end slide -------------------------------%}}}
%-------------- start slide -------------------------------%{{{ --
%%ADD THIS SLIDE FOR QR-FACTORIZATION
%%\frame{
%%\begin{example}[continued]
%%\begin{eqnarray*}
%%R & = & UA \\
%%U^{-1} R & = & U^{-1}(UA) \\
%%& = & (U^{-1}U)A \\
%%& = & IA \\
%%& = & A.
%%\end{eqnarray*}
%%\uncover<2->{
%%Thus $A=U^{-1} R$, and
%%\[ Q=U^{-1}=
%%\left[\begin{array}{rr}\vspace*{.02in}
%%\frac{1}{3} & 0 \\
%%\frac{2}{3} & -1
%%\end{array}\right]^{-1}
%%=\left[\begin{array}{rr}
%%3 & 0 \\
%%2 & -1
%%\end{array}\right].
%%\]}
%%\end{example}
%%}
%-------------- end slide -------------------------------%}}}
%-------------- start slide -------------------------------%{{{ 14
\frame{
\begin{example}[ A Matrix as a product of elementary matrices ]
    Let
    \[ A= \left[\begin{array}{rrr}
	    1 & 2 & -4 \\
	    -3 & -6 & 13 \\
    0 & -1 & 2 \end{array}\right].\]
    \pause
    Suppose we do row operations to put $A$ in reduced row-echelon form,
    and write down the corresponding elementary matrices.
    \pause
    \[
	\left[\begin{array}{rrr}
		1 & 2 & -4 \\
		-3 & -6 & 13 \\
		0 & -1 & 2
	\end{array}\right]
	\pause
	\stackrel{\longrightarrow}{_{E_1}}
	\left[\begin{array}{rrr}
		1 & 2 & -4 \\
		0 & 0 & 1 \\
		0 & -1 & 2
	\end{array}\right]
	\pause
	\stackrel{\longrightarrow}{_{E_2}}
	\left[\begin{array}{rrr}
		1 & 2 & -4 \\
		0 & -1 & 2\\
		0 & 0 & 1
	\end{array}\right]
	\pause
	\stackrel{\longrightarrow}{_{E_3}}
    \]
    \[
	\left[\begin{array}{rrr}
		1 & 2 & -4 \\
		0 & 1 & -2\\
		0 & 0 & 1
	\end{array}\right]
	\pause
	\stackrel{\longrightarrow}{_{E_4}}
	\left[\begin{array}{rrr}
		1 & 0 & 0 \\
		0 & 1 & -2\\
		0 & 0 & 1
	\end{array}\right]
	\pause
	\stackrel{\longrightarrow}{_{E_5}}
	\left[\begin{array}{rrr}
		1 & 0 & 0 \\
		0 & 1 & 0\\
		0 & 0 & 1
	\end{array}\right]
    \]
    \pause
    Notice that the reduced row-echelon form of $A$ equals $I_3$.
    Now find the matrices $E_1, E_2, E_3, E_4$ and $E_5$.
\end{example}
}
%-------------- end slide -------------------------------%}}}
%-------------- start slide -------------------------------%{{{ 15
\frame{
\begin{example}[continued]
    \[ E_1 =\left[\begin{array}{rrr}
    1 & 0 & 0 \\
    3 & 1 & 0\\
    0 & 0 & 1
    \end{array}\right],
    \pause
    E_2 =\left[\begin{array}{rrr}
    1 & 0 & 0 \\
    0 & 0 & 1\\
    0 & 1 & 0\\
    \end{array}\right],
    \pause
    E_3 =\left[\begin{array}{rrr}
    1 & 0 & 0 \\
    0 & -1 & 0\\
    0 & 0 & 1
    \end{array}\right]
    \]
    \pause
    \[
    E_4 =\left[\begin{array}{rrr}
    1 & -2 & 0 \\
    0 & 1 & 0\\
    0 & 0 & 1
    \end{array}\right],
    \pause
    E_5 =\left[\begin{array}{rrr}
    1 & 0 & 0 \\
    0 & 1 & 2\\
    0 & 0 & 1
    \end{array}\right]
    \]
    \pause
    It follows that
    \begin{eqnarray*}
    (E_5(E_4(E_3(E_2(E_1A))))) & = & I\\
    (E_5E_4E_3E_2E_1)A & = & I
    \end{eqnarray*}
    and therefore
    \[ A^{-1}=E_5E_4E_3E_2E_1 \]
\end{example}
}
%-------------- end slide -------------------------------%}}}
%-------------- start slide -------------------------------%{{{ 16
\frame{
\begin{example}[continued]
    Since $A^{-1}=E_5E_4E_3E_2E_1$,
    \begin{eqnarray*}
    A^{-1}        & = & E_5E_4E_3E_2E_1        \\
    (A^{-1})^{-1} & = & (E_5E_4E_3E_2E_1)^{-1} \\
    A             & = & E_1^{-1}E_2^{-1}E_3^{-1}E_4^{-1}E_5^{-1}
    \end{eqnarray*}
\end{example}
\vfill
\pause
\begin{emptytitle}
    This example illustrates the following result.
\end{emptytitle}
\vfill
\pause
\begin{theorem}
    Let $A$ be an $n \times n$ matrix. Then, $A^{-1}$ exists if and only if $A$ can be written as the product of elementary matrices.
\end{theorem}
}
%-------------- end slide -------------------------------%}}}
%-------------- start slide -------------------------------%{{{ 17
\begin{frame}[fragile]
    \begin{example}[ revisited  -- Matrix inversion algorithm]
	\begin{align*}
	    \left[\begin{array}{r|r}
		    A & I
	    \end{array}\right]
	    & =
	    \left[
		\begin{array}{rrr}
		    1 & 2 & -4 \\
		    -3 & -6 & 13 \\
		    0 & -1 & 2
		\end{array}\middle|\quad I \quad\right]\\[1em]
	    E_1 \left[\begin{array}{r|r}
		    A & I
	    \end{array}\right]
	    &=
	    \left[
		\begin{array}{rrr}
		    1 & 2 & -4 \\
		    0 & 0 & 1 \\
		    0 & -1 & 2
		\end{array}\middle| \quad E_1\quad \right]
	    &&=
	    \left[
		\begin{array}{rrr}
		    1 & 2 & -4 \\
		    0 & 0 & 1 \\
		    0 & -1 & 2
		\end{array}
		\middle|
		\begin{array}{rrr}
		    1 & 0 & 0 \\
		    3 & 1 & 0\\
		    0 & 0 & 1
		\end{array}
	    \right]
	    \\[1em]
	    E_2E_1 \left[\begin{array}{r|r}
		    A & I
	    \end{array}\right]
	    &=
	    \left[
		\begin{array}{rrr}
		    1 & 2 & -4 \\
		    0 & -1 & 2\\
		    0 & 0 & 1
		\end{array}\middle| \quad E_2E_1\quad\right]
	    &&=
	    \left[
		\begin{array}{rrr}
		    1 & 2 & -4 \\
		    0 & -1 & 2\\
		    0 & 0 & 1
		\end{array}
		\middle|
		\begin{array}{rrr}
		    1 & 0 & 0 \\
		    0 & 0 & 1\\
		    3 & 1 & 0
		\end{array}
	    \right]
	\end{align*}

    \end{example}
\end{frame}
%-------------- end slide -------------------------------%}}}
%-------------- start slide -------------------------------%{{{ 18
\begin{frame}[fragile]
    \begin{example}[ continued ]
     \begin{align*}
	    E_3E_2E_1
	    [~A~|~I~]
	    &=
	    \left[
		\begin{array}{rrr}
		    1 & 2 & -4 \\
		    0 & 1 & -2\\
		    0 & 0 & 1
		\end{array}\middle| E_3E_2E_1\right]
	    &&=
	    \left[
		\begin{array}{rrr}
		    1 & 2 & -4 \\
		    0 & 1 & -2\\
		    0 & 0 & 1
		\end{array}
		\middle|
		\begin{array}{rrr}
		    1 & 0 & 0 \\
		    0 & 0 & -1\\
		    3 & 1 & 0
		\end{array}
	    \right]
	    \\[1em]
	    E_4E_3E_2E_1
	    [~A~|~I~]
	    &=
	    \left[
		\begin{array}{rrr}
		    1 & 0 & 0 \\
		    0 & 1 & -2\\
		    0 & 0 & 1
		\end{array}\middle|E_4E_3E_2E_1\right]
	    &&=
	    \left[
		\begin{array}{rrr}
		    1 & 0 & 0 \\
		    0 & 1 & -2\\
		    0 & 0 & 1
		\end{array}
	    \middle|
		\begin{array}{rrr}
		    1 & 0 & 2 \\
		    0 & 0 & -1\\
		    3 & 1 & 0
		\end{array}
	    \right]
	    \\[1em]
	\hspace{-8em}
	    E_5E_4E_3E_2E_1
	    [~A~|~I~]
	    &=
	    \left[
		\begin{array}{rrr}
		    1 & 0 & 0 \\
		    0 & 1 & 0\\
		    0 & 0 & 1
		\end{array}
		\middle|
		    E_5E_4E_3E_2E_1
	    \right]
	    &&=
	    \left[
		\begin{array}{rrr}
		    1 & 0 & 0 \\
		    0 & 1 & 0\\
		    0 & 0 & 1
		\end{array}
		\middle|
		\begin{array}{rrr}
		    1 & 0 & 2 \\
		    6 & 2 & -1\\
		    3 & 1 & 0
		\end{array}
	    \right]
	\end{align*}
	\bigskip
	\begin{align*}
	    A^{-1} = E_5E_4E_3E_2E_1 =
	    \left[
		\begin{array}{rrr}
		    1 & 0 & 2 \\
		    6 & 2 & -1\\
		    3 & 1 & 0
		\end{array}
	    \right]
	\end{align*}
    \end{example}
\end{frame}
%-------------- end slide -------------------------------%}}}
%-------------- start slide -------------------------------%{{{ 19
\frame{
\begin{problem}
    Express $A=
    \left[\begin{array}{rr}
    4 & 1 \\
    -3 & 2
    \end{array}\right]$ as a product of elementary matrices.
\end{problem}
\vfill
\pause
\begin{solution}
    \[
    \left[\begin{array}{rr}
    4 & 1 \\ -3 & 2
    \end{array}\right]
    \pause
    \stackrel{\longrightarrow}{_{E_1}}
    \left[\begin{array}{rr}
    1 & 3 \\ -3 & 2
    \end{array}\right]
    \pause
    \stackrel{\longrightarrow}{_{E_2}}
    \left[\begin{array}{rr}
    1 & 3 \\ 0 & 11
    \end{array}\right]
    \pause
    \stackrel{\longrightarrow}{_{E_3}}
    \left[\begin{array}{rr}
    1 & 3 \\ 0 & 1
    \end{array}\right]
    \pause
    \stackrel{\longrightarrow}{_{E_4}}
    \left[\begin{array}{rr}
    1 & 0 \\ 0 & 1
    \end{array}\right]
    \]
    \pause
    with
    \[
    E_1 =\left[\begin{array}{rr}
    1 & 1 \\ 0 & 1
    \end{array}\right],
    \pause
    E_2 =\left[\begin{array}{rr}
    1 & 0 \\ 3 & 1
    \end{array}\right],
    \pause
    E_3 =\left[\begin{array}{rr}
    1 & 0 \\ 0 & \frac{1}{11}
    \end{array}\right],
    \pause
    E_4 =\left[\begin{array}{rr}
    1 & -3 \\ 0 & 1
    \end{array}\right]
    \]
    \pause
    Since $E_4E_3E_2E_1A=I$, $A^{-1}=E_4E_3E_2E_1$, and hence
    \[ A= E_1^{-1} E_2^{-1} E_3^{-1} E_4^{-1} \]
\end{solution}
}
%-------------- end slide -------------------------------%}}}
%-------------- start slide -------------------------------%{{{ 20
\frame{
\begin{solution}[continued]
    Therefore,
    \[ A =
    \left[\begin{array}{rr}
	1 & 1 \\
	0 & 1
    \end{array}\right]^{-1}
    \left[\begin{array}{rr}
	1 & 0 \\
	3 & 1
    \end{array}\right]^{-1}
    \left[\begin{array}{rr}
	1 & 0 \\
	0 & 1/11
    \end{array}\right]^{-1}
    \left[\begin{array}{rr}
	1 & -3 \\
	0 & 1
    \end{array}\right]^{-1}
    \]
    \pause
    i.e.,
    \[ A =
    \left[\begin{array}{rr}
	1 & -1 \\
	0 & 1
    \end{array}\right]
    \left[\begin{array}{rr}
	1 & 0 \\
	-3 & 1
    \end{array}\right]
    \left[\begin{array}{rr}
	1 & 0 \\
	0 & 11
    \end{array}\right]
    \left[\begin{array}{rr}
	1 & 3 \\
	0 & 1
    \end{array}\right]
    \]
    \myQED
\end{solution}
}
%-------------- end slide -------------------------------%}}}
%-------------- start slide -------------------------------%{{{ 21
\frame{
\begin{emptytitle}
    One result that we have assumed in all our work involving
    reduced row-echelon matrices is the following.
\end{emptytitle}
\pause
\vfill
\begin{theorem}[ Uniqueness of the Reduced Echelon Form ]
    If $A$ is an $m\times n$ matrix and $R$ and $S$ are reduced row-echelon forms of $A$, then $R=S$.
\end{theorem}
\pause
\vfill
\begin{remark}
    \alert{This theorem ensures that the reduced row-echelon form of
    a matrix is unique,}
    \pause
    and its proof follows from the results about elementary matrices.
\end{remark}
}
%-------------- end slide -------------------------------%}}}
\section[\textcolor{yellow}{}]{\textcolor{yellow}{Smith Normal Form}}
%-------------- start slide -------------------------------%{{{ 22
\begin{frame}[fragile]
\frametitle{Smith Normal Form}
\pause
\begin{definition}
    If $A$ is an $m\times n$ matrix of rank $r$, then the matrix
    $\begin{pmatrix} I_r & 0 \cr 0 & 0 \end{pmatrix}_{m \times n}$
    is called the \alert{Smith normal form} of $A$.
\end{definition}
\vfill
\begin{theorem}
    If $A$ is an $m\times n$ matrix of rank $r$, then there exist invertible
    matrices $U$ and $V$ of size $m \times m$ and $n \times n$, respectively,
    such that
    \begin{align*}
        UAV = \begin{pmatrix} I_r & 0 \cr 0 & 0 \end{pmatrix}_{m \times n}
    \end{align*}
\end{theorem}
\end{frame}
%-------------- end slide -------------------------------%}}}
%-------------- start slide -------------------------------%{{{ 23
\begin{frame}[fragile]
\begin{proofnoend}
    \begin{enumerate}
	\item Apply the elementary row operations:
	     \begin{align*}
		 [A | I_m]
		\stackrel{e.r.o.}{\longrightarrow}
		 [\RRef\left(A\right)| U]
	     \end{align*}
	\item Apply the elementary column operations:
	    \begin{align*}
		\begin{pmatrix} \RRef(A) \cr I_n \end{pmatrix}
		\stackrel{e.c.o.}{\longrightarrow}
	    \begin{pmatrix} \begin{pmatrix} I_r & 0 \cr 0 & 0 \end{pmatrix}_{m \times n} \cr V \end{pmatrix}_{2m \times n}
	    \end{align*}
    \end{enumerate}
    \myQED
\end{proofnoend}
   \vfill
   \begin{remark}
       The elementary column operations above are equivalent to the elementary row operations on the transpose:
	    \begin{align*}
		\left[\RRef(A)^T \middle| I_n \right]
		\stackrel{e.r.o.}{\longrightarrow}
	    \left[\begin{pmatrix} I_r & 0 \cr 0 & 0 \end{pmatrix}_{n \times m} \middle| V^T\right]_{n \times 2m}
	    \end{align*}
   \end{remark}
\end{frame}
%-------------- end slide -------------------------------%}}}
%-------------- start slide -------------------------------%{{{ 24
\begin{frame}[fragile]
\begin{problem}
Find the decomposition of $A=\left[
    \begin{array}{rrr}
	3 & 0  & 1 \\
	2 & -1 & 0
    \end{array}\right]$
    into the Smith normal form: $A= \widetilde{U}N \widetilde{V}$, where $N$ is the Smith normal form of $A$
    and $\widetilde{U}, \widetilde{V}$ are some invertible matrices.
\end{problem}
\pause
\vfill
\begin{solution}
    We have seen that
    \[
	[A|I_2]= \left[\begin{array}{rrr|rr}
	    3 & 0  & 1 & 1 & 0 \\
	    2 & -1 & 0 & 0 & 1
	\end{array}\right]
	\rightarrow
	\left[\begin{array}{rrr|rr}
	    1 & 0 & 1/3 & 1/3 & 0 \\
	    0 & 1 & 2/3 & 2/3 & -1
	\end{array}\right]
	=
	[\RRef(A) | U]
    \]
    Now,
    \begin{align*}
	\begin{pmatrix} \RRef(A)^T~\bigg|~ I_3 \end{pmatrix}
	=
	\left[
	\begin{array}{cc|ccc}
	    1           & 0           & 1 & 0 & 0\\
	    0           & 1           & 0 & 1 & 0\\
	    \frac{1}{3} & \frac{2}{3} & 0 & 0 & 1\\
	\end{array}\right]
	\rightarrow
	\left[
	\begin{array}{cc|ccc}
	    1 & 0 & 1            & 0            & 0\\
	    0 & 1 & 0            & 1            & 0\\
	    0 & 0 & -\frac{1}{3} & -\frac{2}{3} & 1\\
	\end{array}\right]
    =
    \left[N^T \middle| V^T\right]
    \end{align*}
\end{solution}
\end{frame}
%-------------- end slide -------------------------------%}}}
%-------------- start slide -------------------------------%{{{ 25
\begin{frame}[fragile]
\begin{solution}[Continued]
    Hence, we find $N = UAV$, namely,
    \begin{align*}
	\begin{pmatrix}
	    1 & 0 & 0\\
	    0 & 1 & 0\\
	\end{pmatrix}
	 =
	\begin{pmatrix}
	    1/3 & 0 \\
	    2/3 & -1
	\end{pmatrix}
	\begin{bmatrix}
	    3 & 0 & 1 \\
	    2 & -1 & 0
	\end{bmatrix}
	\begin{pmatrix}
	    1 & 0 & -1/3\\
	    0 & 1 & -2/3\\
	    0 & 0 & 1\\
	\end{pmatrix}
    \end{align*}
    Finally, since $U$ and $V$ are invertible, we see that
    \begin{align*}
	A = U^{-1} N V^{-1},
    \end{align*}
    namely,
    \begin{align*}
	A=
	\begin{bmatrix}
	    3 & 0 & 1 \\
	    2 & -1 & 0
	\end{bmatrix}
	 &=
	\begin{pmatrix}
	    1/3 & 0 \\
	    2/3 & -1
	\end{pmatrix}^{-1}
	\begin{pmatrix}
	    1 & 0 & 0\\
	    0 & 1 & 0\\
	\end{pmatrix}
	\begin{pmatrix}
	    1 & 0 & -1/3\\
	    0 & 1 & -2/3\\
	    0 & 0 & 1\\
	\end{pmatrix}^{-1}\\
	 &=\begin{pmatrix}
	    3 & 0 \\
	    2 & -1
	\end{pmatrix}
	\begin{pmatrix}
	    1 & 0 & 0\\
	    0 & 1 & 0\\
	\end{pmatrix}
	\begin{pmatrix}
	    1 & 0 & 1/3\\
	    0 & 1 & 2/3\\
	    0 & 0 & 1\\
	\end{pmatrix}\\
	 &=
	 \widetilde{U} N \widetilde{V}.
    \end{align*}
    \myQED
\end{solution}
\end{frame}
%-------------- end slide -------------------------------%}}}
\end{document}

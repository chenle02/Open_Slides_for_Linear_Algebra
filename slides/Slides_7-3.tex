%%%%%%%%%%%%%%%%%%%%% {{{
%%Options for presentations (in-class) and handouts (e.g. print).
\documentclass[pdf,9pt]{beamer}


%%%%%%%%%%%%%%%%%%%%%%
%Change this for different slides so it appears in bar
\usepackage{authoraftertitle}
\date{Chapter 7. Linear Transformations \\ \S  7-3. Isomorphisms and Composition}

%%%%%%%%%%%%%%%%%%%%%%
%% Upload common style file
\usepackage{LyryxLAWASlidesStyle}

\begin{document}

%%%%%%%%%%%%%%%%%%%%%%%
%% Title Page and Copyright Common to All Slides

%Title Page
\input frontmatter/titlepage.tex

%LOTS Page
\input frontmatter/lyryxopentexts.tex

%Copyright Page
\input frontmatter/copyright.tex

%%%%%%%%%%%%%%%%%%%%%%%%% }}}
%-------------- start slide -------------------------------%{{{ 2

\begin{frame}[fragile]
   \tableofcontents
\end{frame}
%-------------- end slide -------------------------------%}}}
\section[\textcolor{yellow}{}]{\textcolor{yellow}{What is isomorphism?}}
%-------------- start slide -------------------------------%{{{ 3
\frame{
\frametitle{What is an isomorphism?}
\pause
\begin{example}
    ${\cal P}_1 = \left\{ax+b ~|~ a,b\in\RR\right\}$, has
    addition and scalar multiplication defined as follows:
    \begin{eqnarray*}
	(a_1x + b_1) + (a_2x+b_2) & = & (a_1 + a_2) x + (b_1 +b_2), \\
	k(a_1x + b_1)             & = & (ka_1)x + (kb_1),
    \end{eqnarray*}
    for all $(a_1x + b_1), (a_2x+b_2)\in{\cal P}_1$ and $k\in\RR$.
    \bigskip

    The role of the variable $x$ is to distinguish $a_1$ from
    $b_1$, $a_2$ from $b_2$, $(a_1+a_2)$ from $(b_1+b_2)$, and
    $(ka_1)$ from $(kb_1)$.
\end{example}
}
%-------------- end slide -------------------------------%}}}
%-------------- start slide -------------------------------%{{{ 4
\begin{frame}[fragile]
\begin{example}[continued]
    This can be accomplished equally well by using vectors in $\RR^2$.
    \[ \RR^2=\left\{\left. \left[\begin{array}{c} a \\ b \end{array}\right]
    ~\right|~ a,b\in\RR \right\}\]
    where addition and scalar multiplication are defined as follows:
    \[
	\left[\begin{array}{c}  a_1     \\ b_1     \end{array}\right]
	+\left[\begin{array}{c} a_2     \\ b_2     \end{array}\right]
	=\left[\begin{array}{c} a_1+a_2 \\ b_1+b_2 \end{array}\right],~
	k\left[\begin{array}{c} a_1     \\ b_1     \end{array}\right]
	=\left[\begin{array}{c} ka_1    \\ kb_1    \end{array}\right]
    \]
    for all $\left[\begin{array}{c} a_1 \\ b_1 \end{array}\right],
    \left[\begin{array}{c} a_2 \\ b_2 \end{array}\right]
    \in\RR^2$ and $k\in\RR$.
 \end{example}
\end{frame}
%-------------- end slide -------------------------------%}}}
%-------------- start slide -------------------------------%{{{ 5
\begin{frame}[fragile]
\begin{definition}
  Let $V$ and $W$ be vector spaces, and $T:V\rightarrow W$ a
  linear transformation.
  $T$ is an \alert{isomorphism} if and only if $T$ is both
  one-to-one and onto (i.e., $\ker(T)=\{\bf{0}\}$ and $\im(T)=W$).
  If $T:V\to W$ is an isomorphism, then the vector spaces
  $V$ and $W$ are said to be \alert{isomorphic}, and
  we write \alert{$V\cong W$}.
\end{definition}
\vfill
\only<1>{
    \begin{center}
    \begin{tikzpicture}[scale=0.75]
    %First image
    \draw (0,0) [dkgreenvect,thick] ellipse (0.95cm and 1.55cm);
    \draw (3,0) [dkgreenvect,thick] ellipse (1.55cm and 0.95cm);
    \draw (0,0) [dkgreenvect,thick] ellipse (0.58cm and 0.68cm);
    \node[font=\footnotesize,text=white](kerT) at (0,0) {$\text{ker}(T)$};
    \draw[dkgreenvect,->,thick](0.05,0.68)--(2.98,0.04);
    \draw[dkgreenvect,->,thick](0.05,-0.68)--(2.98,-0.04);
    \draw[-latex, dkgreenvect, thick](0.7,1.05)..controls (0.9,1.15) and (1.55,1.15)..(2.15,0.8) node[above,midway,font=\footnotesize,text=white]{$T$};
    \node[font=\footnotesize,text=white] at (0,-1.15) {$V$};
    \node[font=\footnotesize,text=white] at (3.5,0) {$W$};
    \draw[dkgreenvect,thick] (3,0) circle (1.5pt) node[below,font=\footnotesize,text=white]{$\vec{0}$};
    \end{tikzpicture}
    %Second image
    \qquad
    \begin{tikzpicture}[scale=0.75]
    \draw (0,-4) [dkgreenvect,thick] ellipse (0.95cm and 1.55cm);
    \draw (3,-4) [dkgreenvect,thick] ellipse (1.55cm and 0.95cm);
    \draw (2.75,-4) [dkgreenvect,thick] ellipse (0.58cm and 0.68cm);
    \node[font=\footnotesize,text=white](imT) at (2.75,-4) {$\text{im}(T)$};
    \node[font=\footnotesize,text=white] at (0,-4) {$V$};
    \node[font=\footnotesize,text=white] at (3.95,-4) {$W$};
    \draw[dkgreenvect,-latex,thick](0.95,-4)--(1.48,-4) node[above,midway,font=\footnotesize,text=white]{$T$};
    \draw[dkgreenvect,->,thick](0.11,-2.46)--(2.85,-3.33);
    \draw[dkgreenvect,->,thick](0.11,-5.54)--(2.85,-4.67);
    \end{tikzpicture}
    \bigskip

    General linear transformation $T$
    \end{center}
}
\only<2>{
    \begin{center}
    \begin{tikzpicture}[scale=0.75]
    %First image
    \draw (0,0) [dkgreenvect,thick] ellipse (0.95cm and 1.55cm);
    \draw (3,0) [dkgreenvect,thick] ellipse (1.55cm and 0.95cm);
    % \draw (0,0) [dkgreenvect,thick] ellipse (0.58cm and 0.68cm);
    \node[font=\footnotesize,text=white](kerT) at (0,0.3) {$\text{ker}(T)=\{\vec{0}\}$};
    % \draw[dkgreenvect,->,thick](0.05,0.68)--(2.98,0.04);
    \draw[dkgreenvect,->,thick](0.,-0)--(2.98,-0);
    \draw[-latex, dkgreenvect, thick](0.7,1.05)..controls (0.9,1.15) and (1.55,1.15)..(2.15,0.8) node[above,midway,font=\footnotesize,text=white]{$T$};
    \node[font=\footnotesize,text=white] at (0,-1.15) {$V$};
    \node[font=\footnotesize,text=white] at (3.5,0) {$W$};
    \draw[dkgreenvect,thick] (3,0) circle (1.5pt) node[below,font=\footnotesize,text=white]{$\vec{0}$};
    \draw[dkgreenvect,thick] (0,0) circle (1.5pt);
    \end{tikzpicture}
    %Second image
    \qquad
    \begin{tikzpicture}[scale=0.75]
    \draw (0,-4) [dkgreenvect,thick] ellipse (0.95cm and 1.55cm);
    \draw (3,-4) [dkgreenvect,thick] ellipse (1.55cm and 0.95cm);
    % \draw (2.75,-4) [dkgreenvect,thick] ellipse (0.58cm and 0.68cm);
    \node[font=\footnotesize,text=white](imT) at (3,-4) {$\text{im}(T)=W$};
    \node[font=\footnotesize,text=white] at (0,-4) {$V$};
    % \node[font=\footnotesize,text=white] at (3.9,-4) {$W$};
    \draw[dkgreenvect,-latex,thick](0.95,-4)--(1.48,-4) node[above,midway,font=\footnotesize,text=white]{$T$};
    \draw[dkgreenvect,->,thick](0.11,-2.46)--(2.05,-3.33);
    \draw[dkgreenvect,->,thick](0.11,-5.54)--(2.05,-4.67);
    \end{tikzpicture}
    \bigskip

    Isomorphism $T$
    \end{center}
}
\end{frame}
%-------------- end slide -------------------------------%}}}
%-------------- start slide -------------------------------%{{{ 6
\frame{
\begin{example}
    The identity operator on any vector space is an isomorphism.
\end{example}
\pause
\vfill
\begin{example}
    $T:{\cal P}_n\to\RR^{n+1}$ defined by
    % \vspace*{-.3in}

    \[ T(a_0 + a_1x + a_2x^2 +\cdots + a_nx^n)
    =\left[\begin{array}{c} a_0 \\ a_1 \\ a_2 \\ \vdots \\ a_n\end{array}\right] \]
    for all $a_0 + a_1x + a_2x^2 +\cdots + a_nx^n\in {\cal P}_n$
    is an isomorphism.
    To verify this, prove that \alert{$T$ is a linear transformation}
    that is \alert{one-to-one} and \alert{onto}.
\end{example}
}
%-------------- end slide -------------------------------%}}}
\section[\textcolor{yellow}{}]{\textcolor{yellow}{Proving vector spaces are isomorphic}}
%-------------- start slide -------------------------------%{{{ 7
\frame{
\frametitle{Proving isomorphism of vector spaces}
\pause
\begin{problem}
    Prove that $\bm{M}_{22}$ and $\RR^4$ are isomorphic.
\end{problem}
\pause
\vfill
\begin{proofnoend}
    Let $T:\bm{M}_{22}\to\RR^4$ be defined by
    \[
    T\left[\begin{array}{cc} a &  b \\ c &  d \end{array}\right]
    =\left[\begin{array}{c}  a \\ b \\ c \\ d \end{array}\right]\mbox{ for all }
    \left[\begin{array}{cc}  a &  b \\ c &  d \end{array}\right]
    \in\bm{M}_{22}. \]
    \pause
    It remains to prove that
    \begin{enumerate}
	\item $T$ is a linear transformation;
	\item $T$ is one-to-one;
	\item $T$ is onto.
    \end{enumerate}
\end{proofnoend}
}
%-------------- end slide -------------------------------%}}}
%-------------- start slide -------------------------------%{{{ 8
\frame{
  \begin{solution}[continued -- 1. linear transformation]
    Let $A=\left[\begin{array}{cc} a_1 & a_2 \\ a_3 & a_4 \end{array}\right],
 	 B=\left[\begin{array}{cc} b_1 & b_2 \\ b_3 & b_4 \end{array}\right]\in
    \bm{M}_{22}$ and let $k\in\RR$.
    Then
    \[
	T(A)= \left[\begin{array}{c} a_1\\a_2\\a_3\\a_4 \end{array}\right] \quad\text{and}\quad
	T(B)= \left[\begin{array}{c} b_1\\b_2\\b_3\\b_4 \end{array}\right].\]
      \pause
      \[\Downarrow\]
      \[ \hspace{-2em}
        T(A+B) = T\left[\begin{array}{cc} a_1+b_1 & a_2+b_2 \\ a_3+b_3 & a_4+b_4 \end{array}\right]
               = \left[\begin{array}{c} a_1+b_1\\a_2+b_2\\a_3+b_3\\a_4+b_4 \end{array}\right]
               = \left[\begin{array}{c} a_1\\a_2\\a_3\\a_4 \end{array}\right] + \left[\begin{array}{c} b_1\\b_2\\b_3\\b_4 \end{array}\right]
               = T(A) + T(B)
      \]
    \pause
     \[\Downarrow\]
    \begin{center}
      $T$ preserves addition.
    \end{center}
    \end{solution}
  }
%-------------- end slide -------------------------------%}}}
%-------------- start slide -------------------------------%{{{ 9
\begin{frame}[fragile]
  \begin{solution}[continued -- 1. linear transformation]
    Also
    \[ T(kA) =
	T\left[\begin{array}{cc} ka_1 &  ka_2 \\ ka_3 &  ka_4 \end{array}\right]
	=\left[\begin{array}{c}  ka_1 \\ ka_2 \\ ka_3 \\ ka_4 \end{array}\right]
	=k\left[\begin{array}{c} a_1  \\ a_2  \\ a_3  \\ a_4  \end{array}\right]
	=kT(A) \]
    \pause
     \[\Downarrow\]
     \begin{center}
      $T$ preserves scalar multiplication.
     \end{center}
     \bigskip

    Since $T$ preserves addition and scalar multiplication, $T$ is a linear
    transformation.
 \end{solution}
\end{frame}
%-------------- end slide -------------------------------%}}}
%-------------- start slide -------------------------------%{{{ 10
\frame{
\begin{solution}[continued -- 2. One-to-one]
  By definition,
  \begin{eqnarray*}
      \ker(T) & = & \left\{ A\in\bm{M}_{22} ~|~ T(A)=\bm{0} \right\} \\
      & = & \left\{
      \left[\begin{array}{cc} a & b \\ c & d \end{array}\right] ~\left|~ a,b,c,d\in\RR \quad\text{and}\quad \left[\begin{array}{c} a\\b\\c\\d \end{array}\right]
      =\left[\begin{array}{c} 0\\0\\0\\0 \end{array}\right] \right.
      \right\}.
  \end{eqnarray*}
  \pause
  If
  $A=\left[\begin{array}{cc} a & b \\ c & d \end{array}\right]\in\ker T$,
  then $a=b=c=d=0$, and thus $\ker(T) = \{ \bm{0}_{22} \}$.
  \pause
  \[\Downarrow\]
  \begin{center}
  $T$ is one-to-one.
  \end{center}
\end{solution}
}
%-------------- end slide -------------------------------%}}}
%-------------- start slide -------------------------------%{{{ 11
\frame{
\begin{solution}[continued -- 3. Onto]
    Let
    \[ X=\left[\begin{array}{c} x_1\\x_2\\x_3\\x_4 \end{array}\right]\in\RR^4,\]
    and
    define matrix $A\in\bm{M}_{22}$ as follows:
    \[ A=\left[\begin{array}{cc} x_1 & x_2 \\ x_3 & x_4 \end{array}\right].\]
    \pause
    Then $T(A)=X$, and therefore $T$ is onto.
    \bigskip

    Finally, since $T$ is a linear transformation that is one-to-one and onto, $T$ is
    an isomorphism.
    \pause
    Therefore, $\bm{M}_{22}$ and $\RR^4$ are isomorphic vector spaces.
    \myQED
\end{solution}
}
%-------------- end slide -------------------------------%}}}
%-------------- start slide -------------------------------%{{{ 12
\frame{
\begin{example}[ Other isomorphic vector spaces ]
    \begin{enumerate}
	\item For all integers $n\geq 0$, ${\cal P}_n\cong \RR^{n+1}$.\\
	\item For all integers $m$ and $n$, $m,n\geq 1$, ${\bm M}_{mn} \cong \RR^{m\times n}$.\\
	\item For all integers $m$ and $n$, $m,n\geq 1$, ${\bm M}_{mn} \cong {\cal P}_{mn-1}$.
    \end{enumerate}
    You should be able to define appropriate linear transformations
    and prove each of these statements.
\end{example}
}
%-------------- end slide -------------------------------%}}}
\section[\textcolor{yellow}{}]{\textcolor{yellow}{Characterizing isomorphisms}}
%-------------- start slide -------------------------------%{{{ 13
\frame{
\frametitle{Characterizing isomorphisms}
\pause
\begin{theorem}
    Let $V$ and $W$ be finite dimensional vector spaces
    and $T:V\to W$ a linear transformation.
    The following are equivalent.
    \begin{enumerate}
	\item $T$ is an isomorphism.
	\item If $\{ \vec{b}_1, \vec{b}_2, \ldots, \vec{b}_n\}$ is
	    any basis of $V$, then
	    $\{ T(\vec{b}_1), T(\vec{b}_2), \ldots, T(\vec{b}_n)\}$ is
	    a basis of $W$.
	\item There exists a basis $\{ \vec{b}_1, \vec{b}_2, \ldots, \vec{b}_n\}$
	    of $V$ such that $\{ T(\vec{b}_1), T(\vec{b}_2), \ldots, T(\vec{b}_n)\}$
	    is a basis of $W$.
    \end{enumerate}
\end{theorem}
\pause
\vfill
\begin{proofnoend}
    (1) $\Rightarrow$ (2): This is because
    \begin{itemize}
        \item[-] One-to-one linear transformations preserve independent sets.
        \item[-] Onto linear transformations preserve spanning sets.
    \end{itemize}
    (2)  $\Rightarrow$ (3) is trivial.\\
\end{proofnoend}
}
%-------------- end slide -------------------------------%}}}
%-------------- start slide -------------------------------%{{{ 1
\begin{frame}[fragile]
    \begin{proofnoend}[Continued]
    (3) $\Rightarrow$ (1). We need to prove that $T$ is both onto and one-to-one.
    \bigskip

    If $T(\vec{v}) = \vec{0}$, write $\vec{v} =
    v_{1}\vec{b}_{1} + \cdots + v_{n}\vec{b}_{n}$ where each $v_{i}$ is in $\RR$.
    Then
    \begin{equation*}
        \vec{0} = T(\vec{v}) = v_{1}T(\vec{b}_{1}) + \cdots + v_{n}T(\vec{b}_{n})
    \end{equation*}
    so $v_{1} = \cdots = v_{n} = 0$ by (3). Hence $\vec{v} = \vec{0}$, so $\func{ker }T = \{\vec{0}\}$ and $T$ is one-to-one.

    \bigskip
    To show that $T$ is onto, let $\vec{w}$ be any vecor in $W$. By (3) there exist $w_{1}, \dots, w_{n}$ in $\RR$ such that
    \begin{equation*}
        \vec{w} = w_{1}T(\vec{b}_1) + \cdots + w_{n}T(\vec{b}_n) = T(w_{1}\vec{b}_1 + \cdots + w_n\vec{b}_n)
    \end{equation*}
    Thus $T$ is onto.
    \myQED
   \end{proofnoend}
\end{frame}
%-------------- end slide -------------------------------%}}}
%-------------- start slide -------------------------------%{{{ 14
\frame{
\begin{emptytitle}
Suppose $V$ and $W$ are finite dimensional
vector spaces with $\dim(V)=\dim(W)$, and let
\[
\{ \vec{b}_1, \vec{b}_2, \ldots, \vec{b}_n \} \quad\text{and}\quad
\{ \vec{f}_1, \vec{f}_2, \ldots, \vec{f}_n \} \]
be bases of $V$ and $W$ respectively.
\pause
Then $T:V\to W$ defined by
\[ T(\vec{b}_i) = \vec{f}_i \mbox{ for } 1\leq k\leq n\]
is a \alert{linear transformation}
that maps a basis of $V$ to a basis of $W$.
By the previous {\bf Theorem}, $T$ is an isomorphism.
\bigskip
\pause

Conversely, if $V$ and $W$ are isomorphic and
$T:V\to W$ is an isomorphism,
then (by the previous {\bf Theorem})
for any basis $\{ \vec{b}_1, \vec{b}_2, \ldots, \vec{b}_n \}$
of $V$,
$\{ T(\vec{b}_1), T(\vec{b}_2), \ldots, T(\vec{b}_n)\}$ is
a basis of $W$,
implying that $\dim(V)=\dim(W)$.
\medskip

This proves the next theorem.
\end{emptytitle}
}
%-------------- end slide -------------------------------%}}}
%-------------- start slide -------------------------------%{{{ 15
\frame{
\begin{theorem}
Finite dimensional vector spaces $V$ and $W$ are isomorphic
if and only if $\dim(V) =\dim(W)$.
\end{theorem}
\pause
\vfill
\begin{corollary}
If $V$ is a vector space with $\dim(V)=n$, then $V$ is
isomorphic to $\RR^n$.
\end{corollary}
}
%-------------- end slide -------------------------------%}}}
%-------------- start slide -------------------------------%{{{ 16
\frame{
\begin{problem}
Let $V$ denote the set of $2\times 2$ real symmetric
matrices.
Then $V$ is a vector space with dimension three.
Find an isomorphism $T:{\cal P}_2\to V$ with the property
that $T(1)=I_2$ (the $2\times 2$ identity matrix).
\end{problem}
\pause
\vfill
\begin{solution}
\[ V = \left\{ \left.
\left[\begin{array}{rr} a & b \\ b & c\end{array}\right]
~\right|~ a,b,c\in\RR\right\}
= \Span\left\{
\left[\begin{array}{rr} 1 & 0 \\ 0 & 0\end{array}\right],
\left[\begin{array}{rr} 0 & 1 \\ 1 & 0\end{array}\right],
\left[\begin{array}{rr} 0 & 0 \\ 0 & 1\end{array}\right] \right\}.\]
\pause
Let
\[ B=\left\{
\left[\begin{array}{rr} 1 & 0 \\ 0 & 0\end{array}\right],
\left[\begin{array}{rr} 0 & 1 \\ 1 & 0\end{array}\right],
\left[\begin{array}{rr} 0 & 0 \\ 0 & 1\end{array}\right] \right\}.\]
Then $B$ is independent, and $\Span(B)=V$, so $B$ is a basis of $V$.
Also, $\dim(V)=3=\dim({\cal P}_2)$.
\pause
However, we want a basis of $V$ that contains $I_2$.
\end{solution}
}
%-------------- end slide -------------------------------%}}}
%-------------- start slide -------------------------------%{{{ 17
\frame{
\begin{solution}[continued]
Let
\[ B^{\prime} =\left\{
\left[\begin{array}{rr} 1 & 0 \\ 0 & 1\end{array}\right],
\left[\begin{array}{rr} 0 & 1 \\ 1 & 0\end{array}\right],
\left[\begin{array}{rr} 0 & 0 \\ 0 & 1\end{array}\right] \right\}.\]
Since $B^{\prime}$ consists of $\dim(V)$ symmetric independent
matrices, $B^{\prime}$ is a basis of $V$.
Note that $I_2\in B^{\prime}$.
\pause
Define
\[ T(1)=\left[\begin{array}{rr} 1 & 0 \\ 0 & 1\end{array}\right],
T(x)=\left[\begin{array}{rr} 0 & 1 \\ 1 & 0\end{array}\right],
T(x^2)=\left[\begin{array}{rr} 0 & 0 \\ 0 & 1\end{array}\right].\]
Then for all $ax^2 + bx +c\in{\cal P}_2$,
\[ T(ax^2 + bx +c)
=\left[\begin{array}{cc} c & b \\ b & a+c\end{array}\right],\]
and $T(1)=I_2$.
\bigskip

By the previous {\bf Theorem}, $T:{\cal P}_2\to V$ is an isomorphism. \myQED
\end{solution}
}
%-------------- end slide -------------------------------%}}}
%-------------- start slide -------------------------------%{{{ 18
\frame{
\begin{theorem}
Let $V$ and $W$ be vector spaces, and $T:V\rightarrow W$ a linear
transformation.
If $\dim(V)=\dim(W)=n$, then $T$ is an isomorphism if and only if
$T$ is either one-to-one or onto.
\end{theorem}
\pause
\vfill
\begin{proofnoend}
\noindent $({\Rightarrow})$
By definition, an isomorphism is both one-to-one and onto.
\medskip
\pause

\noindent $({\Leftarrow})$
Suppose that $T$ is one-to-one.
Then $\ker(T)=\{\vec{0}\}$, so $\dim(\ker(T))=0$.
By the Dimension Theorem,
\begin{eqnarray*}
\dim(V) & = & \dim(\im(T)) + \dim(\ker(T)) \\
n & = & \dim(\im(T)) + 0
\end{eqnarray*}
so $\dim(\im(T))=n=\dim(W)$.
Furthermore $\im(T)\subseteq W$, so it follows that $\im(T) = W$.
Therefore, $T$ is onto, and hence is an isomorphism.
\end{proofnoend}
}
%-------------- end slide -------------------------------%}}}
%-------------- start slide -------------------------------%{{{ 19
\frame{
\begin{proofnoend}[continued]
    $({\Leftarrow})$
    Suppose that $T$ is onto.
    Then $\im(T) = W$, so $\dim(\im(T))=\dim(W) = n$.
    By the Dimension Theorem,
    \begin{eqnarray*}
	\dim(V) & = & \dim(\im(T)) + \dim(\ker(T)) \\
	n       & = & n + \dim(\ker(T))
    \end{eqnarray*}
    so $\dim(\ker(T))=0$.
    The only vector space with dimension zero is the zero vector
    space, and thus $\ker(T) =\{ \vec{0}\}$.
    Therefore, $T$ is one-to-one, and hence is an isomorphism.
    \myQED
\end{proofnoend}
}
%-------------- end slide -------------------------------%}}}
\section[\textcolor{yellow}{}]{\textcolor{yellow}{Composition of transformations}}
%-------------- start slide -------------------------------%{{{ 20
\frame{
\frametitle{Composition of transformations}
\pause
\begin{definition}
    Let $V, W$ and $U$ be vector spaces, and let
    \[ T:V\to W \quad\text{and}\quad S:W\to U\]
    be linear transformations.
    The \alert{composite} of $T$ and $S$ is
    \[ ST:V\to U \]
    where $(ST)(\vec{v})=S(T(\vec{v}))$ for all $\vec{v}\in V$.
    The process of obtaining $ST$ from $S$ and $T$ is called
    \alert{composition}.
\end{definition}
\vfill
\begin{center}
\begin{tikzpicture}[scale=0.75]
\draw (0,0) [dkgreenvect,thick] ellipse (0.65cm and 0.65cm);
\draw (2,0) [dkgreenvect,thick] ellipse (0.65cm and 0.65cm);
\draw (4,0) [dkgreenvect,thick] ellipse (0.65cm and 0.65cm);
\draw[-latex, dkgreenvect, thick](0,0.2)..controls (0.55,0.5) and (1.3,0.5)..(1.9,0.2) node[above,midway,text=white]{$T$};
\draw[-latex, dkgreenvect, thick](2.1,0.2)..controls (2.55,0.5) and (3.3,0.5)..(4,0.2) node[above,midway,text=white]{$S$};
\node[text=white] at (0.6,-0.85) {$V$};
\node[text=white] at (2.6,-0.85) {$W$};
\node[text=white] at (4.6,-0.85) {$U$};
\end{tikzpicture}
\end{center}
}
%-------------- end slide -------------------------------%}}}
%-------------- start slide -------------------------------%{{{ 21
\frame{
\begin{example}
    Let $S:\bm{M}_{22}\to \bm{M}_{22}$ and $T:\bm{M}_{22}\to \bm{M}_{22}$
    be linear transformations such that
    \[ S(A)=-A^T \quad\text{and}\quad
	T\left[\begin{array}{cc}  a & b \\ c & d \end{array}\right]
	=\left[\begin{array}{cc}  b & a \\ d & c \end{array}\right] \mbox{ for all }
	A=\left[\begin{array}{cc} a & b \\ c & d \end{array}\right]
    \in \bm{M}_{22}.\]
    Then
    \[
	(ST)\left[\begin{array}{cc} a  & b  \\ c  & d  \end{array}\right]
	=S\left[\begin{array}{cc}   b  & a  \\ d  & c  \end{array}\right]
	=\left[\begin{array}{cc}    -b & -d \\ -a & -c \end{array}\right],
    \]
    and
    \[
	(TS)\left[\begin{array}{cc} a  & b  \\ c  & d  \end{array}\right]
	=T\left[\begin{array}{cc}   -a & -c \\ -b & -d \end{array}\right]
	=\left[\begin{array}{cc}    -c & -a \\ -d & -b \end{array}\right].\]
    If $a, b, c$ and $d$ are distinct, then $(ST)(A)\neq (TS)(A)$.
    \medskip

    \alert{This illustrates that, in general, $ST\neq TS$.}
\end{example}
}
%-------------- end slide -------------------------------%}}}
%-------------- start slide -------------------------------%{{{ 22
\frame{
\begin{theorem}
    Let $V,W,U$ and $Z$ be vector spaces and
    \[
	V\stackrel{T}{\rightarrow}
	W\stackrel{S}{\rightarrow}
	U\stackrel{R}{\rightarrow} Z\]
    be linear transformations.
    Then
    \begin{enumerate}
	\item $ST$ is a linear transformation.
	\item $T1_V= T$ and $1_W T = T$.
	\item $(RS)T=R(ST)$.
    \end{enumerate}
\end{theorem}
}
%-------------- end slide -------------------------------%}}}
%-------------- start slide -------------------------------%{{{ 23
\frame{
\begin{problem}[ The composition of onto transformations is onto ]
  Let $V, W$ and $U$ be vector spaces, and let
  \[ V\stackrel{T}{\rightarrow}
  W\stackrel{S}{\rightarrow} U\]
  be linear transformations.
  Prove that if $T$ and $S$ are onto, then $ST$
  is onto.
\end{problem}
\vfill
\pause
\begin{proofnoend}
  Let ${\bm z}\in U$.
  Since $S$ is onto, there exists a vector ${\bm y}\in W$
  such that $S({\bm y})={\bm z}$.
  Furthermore, since $T$ is onto, there exists a vector ${\bm x}\in V$
  such that $T({\bm x})={\bm y}$.
  Thus
  \[ {\bm z} = S({\bm y}) = S(T({\bm x})) = (ST)({\bm x}),\]
  showing that for each ${\bm z}\in U$ there exists and ${\bm x}\in V$
  such that $(ST)({\bm x})={\bm z}$.
  Therefore, $ST$ is onto.
  \myQED
\end{proofnoend}
}
%-------------- end slide -------------------------------%}}}
%-------------- start slide -------------------------------%{{{ 24
\frame{
\begin{problem}[ The composition of one-to-one transformations is one-to-one ]
    Let $V, W$ and $U$ be vector spaces, and let
    \[ V\stackrel{T}{\rightarrow}
    W\stackrel{S}{\rightarrow} U\]
    be linear transformations.
    Prove that if $T$ and $S$ are one-to-one, then $ST$
    is one-to-one.
\end{problem}
\vfill
\pause
\begin{emptytitle}
    The proof of this is left as an exercise.
\end{emptytitle}
}
%-------------- end slide -------------------------------%}}}
\section[\textcolor{yellow}{}]{\textcolor{yellow}{Inverses}}
%-------------- start slide -------------------------------%{{{ 25
\frame{
\frametitle{Inverses}
\pause
\begin{theorem}
    Let $V$ and $W$ be finite dimensional vector spaces,
    and $T:V\to W$ a linear transformation.
    Then the following statements are equivalent.
    \begin{enumerate}
	\item $T$ is an isomorphism.
	\item There exists a linear transformation $S:W\to V$ so that
	    \[ ST=1_V \quad\text{and}\quad TS=1_W.\]
	    In this case, the isomorphism $S$ is uniquely
	    determined by $T$:
	    \[ \mbox{if } \vec{w}\in W \quad\text{and}\quad \vec{w}=T(\vec{v}),
	    \mbox{ then } S(\vec{w})=\vec{v}.\]
    \end {enumerate}
\end{theorem}
\pause
\vfill
\begin{emptytitle}
    Given an isomorphism $T:V\to W$, the unique isomorphism
    satisfying the second condition of the theorem is
    the \alert{inverse} of $T$, and is written \alert{$T^{-1}$}.
\end{emptytitle}
}
%-------------- end slide -------------------------------%}}}
%-------------- start slide -------------------------------%{{{ 26
\frame{
\begin{remark}[ Fundamental Identities (relating $T$ and $T^{-1}$) ]
  If $V$ and $W$ are vector spaces and $T:V\to W$ is an
  isomorphism,
  then $T^{-1}:W\to V$ is a linear transformation
  such that
  \[ (T^{-1}T)(\vec{v})=\vec{v}\quad\text{and}\quad
  (TT^{-1})(\vec{w})=\vec{w}\]
  for each $\vec{v}\in V$, $\vec{w}\in W$.
  Equivalently,
  \[ T^{-1}T=1_V\quad\text{and}\quad TT^{-1}=1_W.\]
\end{remark}
}
%-------------- end slide -------------------------------%}}}
%-------------- start slide -------------------------------%{{{ 27
\frame{
\begin{problem}
  The function $T:{\cal P}_2\to\RR^3$ defined by
  \[ T(a+bx+cx^2)=
  \left[\begin{array}{c} a-c \\ 2b \\ a+c \end{array}\right]
  \mbox{ for all } a+bx+cx^2\in{\cal P}_2\]
  is a linear transformation \alert{(this is left for you to verify)}.
  Does $T$ have an inverse?  If so, find $T^{-1}$.
\end{problem}
}
%-------------- end slide -------------------------------%}}}
%-------------- start slide -------------------------------%{{{ 28
\frame{
\begin{solution}
  Since $\dim({\cal P}_2)=3=\dim(\RR^3)$, it suffices to prove
  that $T$ is either one-to-one or onto.
  \medskip
  \pause

  Suppose $a +bx + cx^2\in\ker(T)$. Then
  \begin{align*}
     \begin{cases}
      a-c=0\\ 2b=0\\ a+c=0
     \end{cases}
     \quad\Longrightarrow\quad
    \begin{cases}
        a=0\\ b=0\\ c=0
    \end{cases}
  \end{align*}
  \pause

  Therefore, $\ker(T)=\{ {\bm 0} \}$, and hence $T$ is one-to-one.
  By our earlier observation, it follows that $T$ is onto, and thus
  is an isomorphism.
\end{solution}
}
%-------------- end slide -------------------------------%}}}
%-------------- start slide -------------------------------%{{{ 29
\frame{
\begin{solution}[continued]
  To find $T^{-1}$, we need to specify
  $T^{-1}\left[\begin{array}{c} p\\ q\\ r \end{array}\right]$
  for any $\left[\begin{array}{c} p\\ q\\ r \end{array}\right]\in\RR^3$.
  \medskip

  Let $a+bx+cx^2\in{\cal P}_2$, and suppose

  \[ T(a+bx+cx^2)=
    \left[\begin{array}{c} p \\ q \\ r \end{array}\right].\]

  By the definition of $T$, $p=a-c$, $q=2b$ and $r=a+c$.
  We now solve for $a, b$ and $c$ in terms of $p, q$ and $r$.
  \[
    \left[\begin{array}{rrr|r}
        1 & 0 & -1 & p \\ 0 & 2 & 0 & q \\ 1 & 0 & 1 & r
    \end{array}\right]
    \rightarrow\cdots\rightarrow
    \left[\begin{array}{rrr|c}
        1 & 0 & 0 & (r+p)/2 \\ 0 & 1 & 0 & q/2 \\ 0 & 0 & 1 & (r-p)/2
    \end{array}\right].
  \]
\end{solution}
}
%-------------- end slide -------------------------------%}}}
%-------------- start slide -------------------------------%{{{ 30
\frame{
\begin{solution}[continued]
    We now have
    $a=\frac{r+p}{2}$, $b=\frac{q}{2}$ and $c=\frac{r-p}{2}$,
    and thus
    \[ T(a+bx+cx^2)
      =\left[\begin{array}{c} p \\ q \\ r \end{array}\right]
      =T\left(\frac{r+p}{2}+\frac{q}{2}x+\frac{r-p}{2}x^2\right)
    \]
    \pause
    Therefore,
    \begin{eqnarray*}
	T^{-1} \left[\begin{array}{c} p\\ q\\ r \end{array}\right]
	& = & T^{-1}\left(T\left(\frac{r+p}{2}+\frac{q}{2}x+\frac{r-p}{2}x^2\right) \right)  \\
	& = & (T^{-1}T)\left(\frac{r+p}{2} + \frac{q}{2}x + \frac{r-p}{2}x^2\right) \\
	& = & \frac{r+p}{2} + \frac{q}{2}x + \frac{r-p}{2}x^2.
    \end{eqnarray*}
    \myQED
\end{solution}
}
%-------------- end slide -------------------------------%}}}
%-------------- start slide -------------------------------%{{{ 31
\frame{
\begin{definition}
    Let $V$ be a vector space with $\dim(V)=n$,
    let $B=\{ \vec{b}_1, \vec{b}_2, \ldots, \vec{b}_n \}$
    be a fixed basis of $V$,
    and let
    $\{ \vec{e}_1, \vec{e}_2, \ldots, \vec{e}_n \}$
    denote the standard basis of $\RR^n$.
    \pause
    We define a transformation $C_B:V\to\RR^n$ by
    \[ C_B(a_1\vec{b}_1 + a_2\vec{b}_2 + \cdots + a_n\vec{b}_n) =
	a_1\vec{e}_1 + a_2\vec{e}_2 + \cdots + a_n\vec{e}_n =
	\left[\begin{array}{c} a_1 \\ a_2 \\ \vdots \\ a_n \end{array}\right].\]

    \pause
    Then $C_B$ is a linear transformation
    such that
    $C_B(\vec{b}_i)=\vec{e}_i$, $1\leq i\leq n$,
    and thus $C_B$ is an isomorphism, called
    \alert{the coordinate isomorphism corresponding to $B$}.
\end{definition}

}
%-------------- end slide -------------------------------%}}}
%-------------- start slide -------------------------------%{{{ 32
\begin{frame}[fragile]
\begin{example}
    Let $V$ be a vector space and
    let $B=\{\vec{b}_1, \vec{b}_2, \ldots,\vec{b}_n\}$
    be a fixed basis of $V$.
    Then $C_B:V\to \RR^n$ is invertible,
    and it is clear that $C_B^{-1}:\RR^n\to V$ is defined by
    % \vspace*{-.1in}

    \[ C_B^{-1}\left[\begin{array}{c} a_1 \\ a_2 \\ \vdots \\ a_n \end{array}\right] =
    a_1\vec{b}_1 + a_2\vec{b}_2 + \cdots + a_n\vec{b}_n \mbox{ for each }
    \left[\begin{array}{c} a_1 \\ a_2 \\ \vdots \\ a_n \end{array}\right] \in\RR^n.  \]
\end{example}
\end{frame}
%-------------- end slide -------------------------------%}}}
\end{document}

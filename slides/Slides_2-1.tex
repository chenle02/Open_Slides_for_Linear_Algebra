%%%%%%%%%%%%%%%%%%%%% {{{
%%Options for presentations (in-class) and handouts (e.g. print).
\documentclass[pdf,9pt]{beamer}
% \documentclass[pdf,9pt]{beamer}


%%%%%%%%%%%%%%%%%%%%%%
%Change this for different slides so it appears in bar
\usepackage{authoraftertitle}
\date{Chapter 2. Matrix Algebra \\ \S 2-1. Matrix Addition, Scalar Multiplication and Transposition}

%%%%%%%%%%%%%%%%%%%%%%
%% Upload common style file
\usepackage{LyryxLAWASlidesStyle}

\begin{document}

%%%%%%%%%%%%%%%%%%%%%%%
%% Title Page and Copyright Common to All Slides

%Title Page
\input frontmatter/titlepage.tex

%LOTS Page
\input frontmatter/lyryxopentexts.tex

%Copyright Page
\input frontmatter/copyright.tex

%%%%%%%%%%%%%%%%%%%%%%%%% }}}
%-------------- start slide -------------------------------%{{{ 2
\begin{frame}[fragile]
   \tableofcontents
\end{frame}
%-------------- end slide -------------------------------%}}}
\section[\textcolor{yellow}{}]{\textcolor{yellow}{Matrices -- Definitions and Basic Properties}}
%-------------- start slide -------------------------------%{{{ 1
\frame{
\frametitle{Matrices -- Definitions and Basic Properties}
\pause
\begin{definition}
    Let $m$ and $n$ be positive integers.
    \begin{itemize}
	\item \alert{An $m\times n$ matrix} is a rectangular array
	    of numbers having $m$ rows and $n$ columns.
	    Such a matrix is said to have \alert{size $m\times n$}.
	    \pause
	\item \alert{A row matrix} (or row) is a $1\times n$ matrix, and
	    \alert{a column matrix} (or column)
	    is an $m\times 1$ matrix.
	    \pause
	\item \alert{A square matrix} is an $n \times n$ matrix.
	    \pause
	\item \alert{The $(i,j)$-entry of a matrix} is the
	    entry in row $i$ and column $j$. For a matrix $A$, the
	    $(i,j)$-entry of $A$ is often written as $a_{ij}$.
    \end{itemize}
\end{definition}
\pause
\vfill
\begin{emptytitle}
    General notation for an $m\times n$ matrix, $A$:
    \vfill
    \[ A=\left[\begin{array}{ccccc}
	    a_{11} & a_{12} & a_{13} & \ldots & a_{1n} \\
	    a_{21} & a_{22} & a_{23} & \ldots & a_{2n} \\
	    a_{31} & a_{32} & a_{33} & \ldots & a_{3n} \\
	    \vdots & \vdots & \vdots &        & \vdots \\
	    a_{m1} & a_{m2} & a_{m3} & \ldots & a_{mn} \\
    \end{array}\right] = \left[ a_{ij} \right] \]
\end{emptytitle}
}
%-------------- end slide -------------------------------%}}}
%-------------- start slide -------------------------------%{{{ 2
\frame{
\begin{remark}[Basic Properties]
\begin{enumerate}
    \item {\bf Equality:} two matrices are equal if and only if they have the same size and the corresponding entries are equal.
	\pause
    \item {\bf Zero Matrix:} an $m\times n$ matrix with all entries equal to zero.
	\pause
    \item {\bf Addition:} matrices must have the same size; add corresponding entries.
	\pause
    \item {\bf Scalar Multiplication:} multiply each entry of the matrix by the scalar.
	\pause
    \item {\bf Negative of a Matrix:} for an $m\times n$ matrix $A$, its negative is denoted $-A$ and $-A=(-1)A$.
	\pause
    \item {\bf Subtraction:} for $m\times n$ matrices $A$ and $B$, $A-B= A+(-1)B$.
\end{enumerate}
\end{remark}
}
%-------------- end slide -------------------------------%}}}
\section[\textcolor{yellow}{}]{\textcolor{yellow}{Matrix Addition}}
%-------------- start slide -------------------------------%{{{ 3
\frame{
\frametitle{Matrix Addition}
\pause

\begin{definition}
Let $A = \leftB a_{ij} \rightB$ and $B = \leftB b_{ij} \rightB$ be two $m \times n$ matrices. Then $A+B = C$ where $C$ is the $m \times n$ matrix $C = \leftB c_{ij} \rightB$ defined by
\[
c_{ij} = a_{ij} + b_{ij}
\]
\end{definition}

\pause

\begin{example}
Let $A = \leftB
\begin{array}{rr}
1 & 3 \\
2 & 5
\end{array}\rightB, B = \leftB
\begin{array}{rr}
0 & -2 \\
6 & 1
\end{array}
\rightB$. Then,
\begin{eqnarray*}
A + B &=&
\leftB
\begin{array}{rr}
1 + 0 & 3 + -2 \\
2 + 6 & 5 + 1
\end{array}
\rightB \\
&=&
\leftB
\begin{array}{rr}
1 & 1 \\
8 & 6
\end{array}
\rightB
\end{eqnarray*}
\end{example}
}
%-------------- end slide -------------------------------%}}}
%-------------- start slide -------------------------------%{{{ 4
\frame{
\begin{theorem}[Properties of Matrix Addition]
Let $A,B$ and $C$ be $m\times n$ matrices.
Then the following properties hold.
\pause
\begin{enumerate}
\item
$A+B=B+A$
\textcolor{blue}{(matrix addition is commutative)}.
\medskip
\pause
\item
$( A+B) +C=A+( B+C)$
\textcolor{blue}{(matrix addition is associative)}.
\medskip
\pause
\item
There exists an $m\times n$ zero matrix, \alert{$0$}, such that
\alert{$A+0=A$}.\\
\textcolor{blue}{(existence of an additive identity)}.
\medskip
\pause
\item
There exists an $m\times n$ matrix \alert{$-A$} such that
\alert{$A+(-A) =0$}.\\
\textcolor{blue}{(existence of an additive inverse)}.
\end{enumerate}
\end{theorem}
}
%-------------- end slide -------------------------------%}}}
\section[\textcolor{yellow}{}]{\textcolor{yellow}{Scalar Multiplication}}
%-------------- start slide -------------------------------%{{{ 5
\frame{
\frametitle{Scalar Multiplication}
\pause
\begin{definition}
    Let $A = \leftB a_{ij} \rightB$ be an $m\times n$ matrix
    and let $k$ be a scalar. Then $kA = \leftB ka_{ij} \rightB$.
\end{definition}
\pause
\vfill
\begin{example}
    Let $A = \leftB
    \begin{array}{rrr}
	2 & 0 & -1 \\
	3 & 1 & -2 \\
	0 & 4 & 5
    \end{array}
    \rightB$.

    \pause
    Then
    \begin{eqnarray*}
    3A &=&
    \leftB
    \begin{array}{rrr}
	3(2) & 3(0) & 3(-1) \\
	3(3) & 3(1) & 3(-2) \\
	3(0) & 3(4) & 3(5)
    \end{array}
    \rightB \\
    &=&
    \leftB
    \begin{array}{rrr}
	6 & 0 & -3 \\
	9 & 3 & -6 \\
	0 & 12 & 15
    \end{array}
    \rightB
    \end{eqnarray*}
\end{example}
}
%-------------- end slide -------------------------------%}}}
%-------------- start slide -------------------------------%{{{ 6
\frame{
\begin{theorem}[Properties of Scalar Multiplication]
    Let $A, B$ be $m\times n$ matrices and let $k, p\in\RR$ (scalars).
    Then the following properties hold.
    \pause
    \begin{enumerate}
	\item $k \left( A+B\right) =k A+ kB$. \\ \textcolor{blue}{(scalar multiplication distributes over matrix addition)}.
	    \medskip
	    \pause
	\item $\left( k +p \right) A= k A+p A$. \\ \textcolor{blue}{(addition distributes over scalar multiplication)}.
	    \medskip
	    \pause
	\item $k \left( p A\right) = \left( k p \right) A$.  \textcolor{blue}{(scalar multiplication is associative)}.
	    \medskip
	    \pause
	\item $1A=A$.  \textcolor{blue}{(existence of a multiplicative identity)}.
    \end{enumerate}
\end{theorem}
}
%-------------- end slide -------------------------------%}}}
%-------------- start slide -------------------------------%{{{ 7
\frame{
\begin{example}
    \[
    2\left[\begin{array}{rr} -1  & 0  \\ 1  & 1  \end{array} \right] +
    4\left[\begin{array}{rr} -2  & 1  \\ 3  & 0  \end{array} \right] -
    \left[\begin{array}{rr}  6   & 8  \\ 1  & -1 \end{array} \right] = \pause
    \left[\begin{array}{rr}  -16 & -4 \\ 13 & 3  \end{array} \right]
    \]
\end{example}
\pause
\vfill
\begin{problem}
    Let $A$ and $B$ be $m\times n$ matrices.
    Simplify the expression
    \[ 2[9(A-B)+7(2B-A)] - 2[3(2B+A) - 2(A+3B) - 5(A+B)] \]
\end{problem}
\pause
\begin{solution}
    \begin{eqnarray*}
        &   & 2[9(A-B)+7(2B-A)] - 2[3(2B+A) - 2(A+3B) - 5(A+B)] \\
        & = & 2(9A-9B+14B-7A) -2(6B+3A-2A-6B-5A-5B)             \\
        & = & 2(2A + 5B) -2(-4A-5B)                             \\
        & = & 12A + 20B
    \end{eqnarray*}
\end{solution}
}
%-------------- end slide -------------------------------%}}}
\section[\textcolor{yellow}{}]{\textcolor{yellow}{The Transpose}}
%-------------- start slide -------------------------------%{{{ 8
\frame{
\frametitle{Matrix Transpose}
\pause
\begin{definition}
    If $A$ is an $m\times n$ matrix, then its \alert{transpose},
    denoted \alert{$A^T$}, is the $n\times m$
    whose $i^{th}$ row is the $i^{th}$ column of $A$, $1\leq i\leq n$;
    i.e., if $A=[a_{ij}]$, then
    \[ A^T= [a_{ij}]^T = [a_{ji}] \]
    i.e., the $(i,j)$-entry of $A^T$ is the $(j,i)$-entry of $A$.
\end{definition}
\pause
\vfill
\begin{theorem}[Properties of the Transpose of a Matrix]
    Let $A$ and $B$ be $m\times n$ matrices,
    $C$ be a $n\times p$ matrix, and $r\in\RR$ a scalar.
    Then
    \pause
    \begin{multicols}{2}
    \begin{enumerate}
	\item $(A^T)^T=A$
	\pause
	\item $(rA)^T = rA^T$
	\pause
	\item $(A+B)^T = A^T + B^T$
	\pause
	\item $(AC)^T=C^T A^T$
    \end{enumerate}
    \end{multicols}
\end{theorem}
\pause
\vfill
\begin{emptytitle}
    To prove each these properties, you only need to compute the $(i,j)$-entries of the
    matrices on the left-hand side and the right-hand side.\pause \alert{ And you can do it!}
\end{emptytitle}
}
%-------------- end slide -------------------------------%}}}
%-------------- start slide -------------------------------%{{{ 9
\frame{
\begin{problem}
    Find the matrix $A$ if
    $\left( A +
    3\left[ \begin{array}{rrr} 1 & -1 & 0 \\ 1 & 2 & 4 \end{array}
    \right] \right)^T =
    \left[ \begin{array}{rr} 2 & 1 \\ 0 & 5 \\ 3 & 8  \end{array}
    \right]$.
\end{problem}
\pause
\vfill
\begin{solution}
\begin{eqnarray*}
    \left[\left( A +
    3\left[ \begin{array}{rrr} 1 & -1 & 0 \\ 1 & 2 & 4 \end{array} \right] \right)^T\right]^T
    & = & \left[ \begin{array}{rr} 2 & 1 \\ 0 & 5 \\ 3 & 8  \end{array} \right]^T \\
    \pause
    A + 3\left[ \begin{array}{rrr} 1 & -1 & 0 \\ 1 & 2 & 4 \end{array}
    \right]
    & = &
    \left[ \begin{array}{rrr} 2 & 0 & 3 \\ 1 & 5 & 8 \end{array}
    \right] \\
    \pause
    A & = &
    \left[ \begin{array}{rrr} 2 & 0 & 3 \\ 1 & 5 & 8 \end{array}
    \right] -
    3\left[ \begin{array}{rrr} 1 & -1 & 0 \\ 1 & 2 & 4 \end{array}
    \right] \\
    \pause
    A & = &
    \left[ \begin{array}{rrr} -1 & 3 & 3 \\ -2 & -1 & -4 \end{array}
    \right]
\end{eqnarray*}
\end{solution}
}
%-------------- end slide -------------------------------%}}}
%-------------- start slide -------------------------------%{{{ 10
\frame{
\begin{definition}
    Let $A=\left[ a_{ij} \right]$ be an $m\times n$ matrix.
    The entries $a_{11}, a_{22}, a_{33},\ldots $ are called the
    \alert{main diagonal} of $A$.
\end{definition}
\pause
\vfill
\begin{definition}[Symmetric Matrices]
    The matrix $A$ is called \alert{symmetric} if and only if
    $A^T=A$.  Note that this immediately implies that $A$ is
    a {\bf square} matrix.
\end{definition}
\vfill
\pause
\begin{examples}
    \[
    \left[ \begin{array}{rr}
    \textcolor{blue}{2} & -3 \\
    -3 & \textcolor{blue}{17}
    \end{array} \right],
    \left[ \begin{array}{rrr}
    \textcolor{blue}{-1} & 0 & 5 \\
    0 & \textcolor{blue}{2} & 11 \\
    5 & 11 & \textcolor{blue}{-3}
    \end{array} \right],
    \left[ \begin{array}{rrrr}
    \textcolor{blue}{0} & 2 & 5 & -1 \\
    2 & \textcolor{blue}{1} & -3 & 0 \\
    5 & -3 & \textcolor{blue}{2} & -7 \\
    -1 & 0 & -7 & \textcolor{blue}{4}
    \end{array} \right]
    \]
    are symmetric matrices, and each
    \textcolor{blue}{is symmetric about its main diagonal}.
\end{examples}
}
%-------------- end slide -------------------------------%}}}
%-------------- start slide -------------------------------%{{{ 12
\frame{
\begin{definition}
    An $n \times n$ matrix $A$ is said to be \alert{skew symmetric} if
    $A^{T} = -A$.
\end{definition}
\vfill
\pause
\begin{example}[Skew Symmetric Matrices]
    \[
	\left[ \begin{array}{rr}
		\textcolor{blue}{0} & 2 \\
		\textcolor{yellow}{-2} & \textcolor{blue}{0}
	\end{array} \right],
	\left[ \begin{array}{rrr}
		\textcolor{blue}{0} & 9 & 4 \\
		\textcolor{yellow}{-9}& \textcolor{blue}{0} & -3 \\
	\textcolor{yellow}{-4} & \textcolor{yellow}{3} & \textcolor{blue}{0} \end{array} \right]
    \]
\end{example}
\pause
\vfill
\begin{problem}
    Show that if $A$ is a square matrix, then $A-A^T$ is skew-symmetric.
\end{problem}
\pause
\vfill
\begin{solution}
    We must show that $(A-A^T)^T=-(A-A^T)$.
    \pause
    Using the properties of matrix addition, scalar multiplication,
    and transposition
    \[ (A-A^T)^T = A^T - (A^T)^T = A^T - A=-(A-A^T).\]
\end{solution}
}
%-------------- end slide -------------------------------%}}}
\end{document}
